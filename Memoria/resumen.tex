\chapter{Resumen}
%\textcolor{red}{Resumen de 100 palabras MÁXIMO}\\

La introducción de código malicioso en imágenes usando esteganografía es un problema muy grave que puede afectar a todos los usuarios de un equipo, ya que si se descarga la imagen y se ejecuta el malware que tiene en su interior el sistema podría quedar infectado. Para solucionarlo, proponemos el uso de un modelo de \ac{IA} con la capacidad suficiente para detectar estos códigos en imágenes. Para ello, hemos cotejado tres posibles modelos que hemos entrenado usando Stegosploit como herramienta principal para introducir malware en imágenes. Como veremos más adelante, el modelo ganador es \textbf{ResNet-34}, y nos servirá para que en un futuro se pueda utilizar en distintas aplicaciones de la misma índole. %cita %cita

%TODO_DONE: Completa esto y luego lo traduces al Inglés abajo.


\vspace{0.5cm}

\textbf{Palabras clave}: \ac{IA}, imágenes, malware, Stegosploit, ResNet-34.
%\textbf{Palabras clave}: cinco palabras como máximo, separadas por comas.

%Introduce hoja en blanco
\newpage
\thispagestyle{empty}
\hspace*{0.5cm}
\newpage

\chapter{Abstract}
%\textcolor{red}{Resumen de 100 palabras MÁXIMO - Es OBLIGATORIO}\\

The introduction of malicious code on images by using stenography is a very serious problem which can affect to every user on a computer, because if the image is downloaded and the malware inside is executed the system could get infected. To solve it, we propose the use of an AI model capable of detecting this hidden codes on images. Thus, we have compared three models we have trained by mainly using the Stegosploit toolkit to introduce the malware on images. As we'll see later, the winning model is \textbf{ResNet-34}, and this one will lead into a future in which it can be used in different applications of the same nature.

\vspace{0.5cm}

\textbf{Keywords}: AI, images, malware, Stegosploit, ResNet-34.
%\textbf{Keywords}: cinco palabras como máximo, separadas por comas.
%Introduce hoja en blanco
\newpage
\thispagestyle{empty}
\hspace*{0.5cm}
\newpage
