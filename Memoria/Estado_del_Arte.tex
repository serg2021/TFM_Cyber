\chapter{Estado del arte}
\label{ch:sota}

En el siguiente capítulo pasaremos a mencionar los artículos que conforman el estado del arte del proyecto para poder hacer una comparativa tanto de los modelos como de las métricas utilizadas y tener una base sobre la que movernos durante la implementación del mismo, la cual veremos más adelante.

\section{Uso de Machine Learning para detectar imágenes con políglotas maliciosos}

%PONER IMAGEN

El objetivo de este proyecto no era otro que hacer una herramienta de #abv:ml para detectar políglotas en imágenes. A continuación, pasaremos a listar y explicar de forma resumida el proceso que realizaron los autores para ello. #cita

Este estudio ha sido realizado tomando como base el artículo de Stegosploit del que también nos hemos servido para este trabajo.

\subsection{Estudiar y construir ataques XSS y CSRF}

En esta fase se procede a ver cómo se puede distribuir a las víctimas las imágenes con el código malicioso en su interior. Para ello, se estudiaron diferentes tipos de ataques sobre los que poder hacerlo, de tal forma que la víctima no se diera cuenta. A continuación, vamos a ver de manera muy superficial en qué consisten:

\subsubsection{Reflected XSS}

Este ataque se basa en la posibilidad de que un atacante modifique una URL de una aplicación con código malicioso. Si esta URL modificada es enviada a un usuario porque éste, por ejemplo, la solicita; el código malicioso se ejecutará en el navegador de la víctima, dentro de la sesión que tenga abierta en la aplicación. #cita

\subsubsection{DOM XSS}

Este ataque, no dista demasiado del anterior, sin embargo tiene otro enfoque: El atacante crea un payload, por ejemplo en la URL de la aplicación, que se ejecuta en el navegador de la víctima debido a una modificación del entorno del DOM usado en el código original de la misma. Así el cliente ejecuta el código del atacante de manera inesperada. De este modo, la página web como tal no cambia, pero en la parte de la víctima, el código contenido en la página se ejecuta de forma diferente por los cambios efectuados en el DOM.

Dicho de otra forma, si por ejemplo se crea un payload en la URL y ésta se envía a la víctima, se modifica el entorno del DOM, el cual detecta el payload en la URL y se ejecuta como un objeto más del código de la aplicación.

Como se puede ver, el resultado es el mismo que en el ataque anterior, no obstante recorre otro camino para conseguirlo. #cita

\subsubsection{Persistent XSS}

Para realizar este ataque hay que inyectar código directamente en un servidor objetivo. De esta forma, cada vez que el usuario pida la información de dicho servidor, obtendrá el código introducido por el atacante. Los servidores objetivo para este tipo de ataques suelen ser bases de datos, un foro de mensajes, un registro de visitas, etc. #cita

\subsubsection{CSRF}

Este tipo de ataques fuerzan a un usuario a ejecutar acciones no deseadas en una aplicación web en la que ya están autenticados. Mediante un email o un chat, un atacante puede enviar a la víctima un enlace para engañarla y ejecutar acciones de la elección del atacante. #cita

\subsection{Creación de políglotas}

Para crear los políglotas con los que explotar el ataque nos basaremos completamente en Stegosploit.

Todo comienza cifrando el código malicioso en los bits de una imagen. Después pasamos a insertar un decodificador en la cabecera de la imagen para crear IMAJS, formando así un políglota. Por último, nos servimos de alguno de los ataques de la fase anterior para atacar a un servidor web con el políglota como recurso.

\subsection{Uso de Machine Learning para detectar imágenes con políglotas maliciosos}

A continuación, pasamos a explicar la última parte de este modelo.

Comenzamos extrayendo 20.000 imágenes del dataset CIFAR-10 para después insertar en la mitad de ellas código malicioso mediante Stegosploit. #cita #cita

Ahora se utiliza el método de árboles de decisión para el dataset completo (80\% de train y 20\% de test). Los árboles de decisión son un método de deep learning usado en problemas de regresión y clasificación; el objetivo es crear un modelo que prediga el valor de una variable aprendiendo mediante simples reglas de decisión inferidas por los datos de entrada. #cita

Los resultados finales vienen representados según una gráfica en la que se utiliza la siguiente métrica: La velocidad de detección de código por densidad de cifrado y profundidad del árbol.

Con esta métrica se quiere comprobar a qué velocidad detecta este método el código cifrado en las imágenes dependiendo de la densidad del cifrado y de la profundidad del árbol; es decir, dependiendo de en cuántos bytes se han modificado los bits y con qué distancia están separados éstos entre sí, y dependiendo de la cantidad de ramificaciones del árbol de decisión según las reglas que ha ido adoptando.

\section{Stegomalware}

Uno de los trabajos más recientes en cuanto a detección de malware en imágenes se refiere, nos presenta una revisión de los modelos de de detección usados durante los últimos tiempos junto con una propuesta de modelo original generada con la base de los mismos. #cita

Dentro del trabajo se extiende una alta gama de contenidos como las herramientas usadas para insertar malware en imágenes mediante esteganografía, el tipo de formato de archivos usados para ello, los algoritmos de esteganografía en imágenes... Para este apartado de la memoria sólo trataremos las técnicas de estegoanálisis usadas para la detección de esteganografía en imágenes, el modelo de detección propuesto por los autores y las métricas usadas durante los experimentos.

\subsection{Técnicas de estegoanálisis}

El estegoanálisis es el acto de determinar si hay datos ocultos en el medio en el que supuestamente están, y el algoritmo esteganográfico usado para introducirlo (además de extraer los datos si los hubiere). A continuación vamos a pasar a explicar cronológicamente, y de forma resumida, el estado del arte de estas técnicas:

\subsubsection{Soluciones de modelos de estegoanálisis enriquecidos y de características basadas en el dominio de imágenes}

Durante las dos últimas décadas se han ido desarrollando diferentes soluciones para abordar el problema de la esteganografía desde este enfoque. Desde un principio la investigación se concentró en detecciones basadas en machine learning y en características de la imagen, seguidas por modelos enriquecidos y por soluciones basadas en clasificadores de conjunto. 

Algunos de los métodos de estegoanálisis como PHARM y GFR siguieron siendo efectivos antes de la aparición del deep learning. Sin embargo, en el momento en el que surge el deep learning, la investigación de estegoanálisis en imágenes convencional se reduce, y es más difícil hacer contribuciones en este ámbito mejorando los métodos anteriormente mencionados. Con el deep learning se da un salto de calidad, y la rapidez con la que se desarrolla es mucho más alta en comparación (incluyendo el estegoanálisis). #cita #cita

Aun con la aparición de esta nueva tecnología, el rendimiento ofrecido por estos métodos sirvió como pie de apoyo para mejorar las soluciones que posteriormente aparecerían gracias al deep learning. Las soluciones de este apartado se evaluaron y compararon con otras usando métricas de error y de precisión de detección. Los investigadores, además, usaron algoritmos de esteganografía estándar como HUGO, UNIWARD o WOW para evaluar la efectividad de las soluciones de estegoanálisis. #cita #cita #cita

Los resultados de la evaluación iban mejorando cada año y, consecuentemente, iban apareciendo más soluciones que probar. Hasta 2015, año de aparición de deep learning, técnicas como GRF o PHARM fueron de lo mejor que hubo. No obstante, ninguno de estos modelos fue evaluado según la detección de malware. Es en este punto donde el deep learning toma partida con el estegoanálisis.

\subsubsection{Modelos de deep learning para el estegoanálisis de imágenes}

Dentro de la lectura de este trabajo, se proponen una buena cantidad de soluciones de deep learning para conseguir la tan ansiada detección de esteganografía en imágenes con unos resultados más prometedores que en el enfoque anterior. Las soluciones principalmente se enfocan en proponer cambios en la capa de preprocesado, en la función de activación, en la disposición de las capas convolucionales y en la fusión de bloques, en la clasificación lineal para capturar elementos esteganográficos.

Al experimentar con las soluciones se ha sacado en claro que hay varios modelos como SR-Net o Zhu-Net, que han conseguido muy buenos resultados en lo que a errores y precisión de detección se refiere. De hecho, estos modelos pueden llegar a ser una referencia para hacer comparativas en el futuro cuando surjan nuevas contribuciones.

Sin embargo, y a pesar de que el campo de deep learning ofrece muchas más posibilidades y caminos que tomar para seguir desarrollándose, todavía no hay trabajos que utilicen esta tecnología para la detección de malware... A continuación veremos la propuesta del modelo de los autores para abarcar un nuevo enfoque en la tecnología de deep learning.

\subsection{Modelo de detección de stegomalware}

El modelo está pensado para aplicarse dentro del ámbito de la empresa y, dependiendo del tamaño de ésta, de la frecuencia de imágenes con esteganografía cotejadas por la misma, el número de archivos multimedia recibidos por la organización y de las operaciones de negocio de ésta; la estructura del modelo variará.

En general, el proceso es como sigue en la imagen:

%PONER IMAGEN

\begin{itemize}
\item Un empleado informa de un archivo de imagen sospechoso en un correo de phishing (por ejemplo), y la imagen será subida a una ubicación de almacenamiento estándar, la cual está aislada del resto de la infraestructura de la aplicación por motivos obvios. Adicionalmente, el equipo de seguridad podrá subir, si precisa, el archivo para analizarlo.
\item El archivo puede ser escaneado por poder tener un posible malware y, además, se puede determinar cuán malicioso mediante herramientas de detección de firma. Estas herramientas indican si la amenaza es conocida o no (si no es conocida, se le pone un identificador para el futuro). #cita
\item Si efectivamente el archivo tiene un malware, marcamos la imagen y realizamos acciones preventivas como aislar la máquina infectada o actualizar el hash de la firma del malware de la imagen en las políticas de seguridad de la organización para bloquear el malware de la imagen y detener la infección de la red.
\item Si por el contrario, no se identifica el archivo como malicioso dentro de este sistema de detección preliminar lo pasamos al siguiente paso: Identificar el tipo de archivo que es y el formato que tiene.
\item Actualmente existen muchos tipos de formato en los archivos multimedia, y uno de los principales retos es identificar el malware en estas condiciones en donde el archivo puede tener cualquier tipo de formato.
\item Si el archivo no es multimedia se seguirá el procedimiento normal de enviarlo a la herramienta anti-malware para su análisis.
\item Si por el contrario, sí se identifica como archivo multimedia, debemos realizar un análisis estructural y estadístico del archivo para la detección del malware. El análisis estructural incluye cambios en la marca de tiempo y en las fechas, propiedades del archivo inusuales como el tamaño del archivo, el checksum y la modificación del contenido, anomalías en el contenido de la cabecera Exif... Para ello, usamos las herramientas de código abierto StegSpy y stego-toolkit. #cita #cita #cita
\item Si hubiera alguna propiedad anómala detectada, se marcaría el archivo como sospechoso para realizar más análisis.
\item Por otro lado, tenemos el análisis estadístico del archivo. Este es realizado para encontrar más pruebas de cuánto de malicioso es el archivo. Las propiedades estadísticas pueden incluir histogramas de bytes y n-gramas del archivo, los cambios de patrón en los píxeles de la imagen o en los frames de un vídeo y cambios en los bit menos significativos de las imágenes. Las herramientas usadas en este tipo de análisis para esta propuesta son: StegExpose y Stegdetect. #cita #cita
\item La respuesta colectiva de los resultados de ambos análisis se combinan y se evalúa de nuevo cuán malicioso es el archivo, o si es sospechoso de tener un malware en su interior.
\item Si el archivo se indica como sospechoso o malicioso, se envía a un entorno de Cuckoo Sandbox para aplicar un análisis de malware dinámico e identificar las características de comportamiento del archivo. #cita
\item Hay una alta probabilidad de que el contenido malicioso oculto en el archivo pueda ser extraído y ejecutado como resultado de unas instrucciones de código embebido.Por ejemplo, un código de shell embebido en el archivo de imagen puede ser ejecutado e intentar conectarse a un servidor remoto para ejecutar comandos maliciosos y exfiltrar datos de la organización.
\item De este modo, basándonos en el comportamiento del malware, tenemos que tomar acciones preventivas en el entorno si el archivo acaba siendo finalmente malicioso.
\item Las acciones preventivas pasan por poder actualizar, de nuevo, la firma del malware para los indicadores de compromiso como direcciones IP, dominios, y otras firmas de código hexadecimal para la detección de malware en el entorno de la infraestructura.
\item Sin embargo, si el archivo se comporta con normalidad durante el análisis dinámico, podemos ignorar el archivo para futuras acciones y podemos seguir la pista del mismo por si se diese el caso de que fuese un falso positivo en el futuro.
\end{itemize} 