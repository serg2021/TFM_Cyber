\chapter{Detección de malware en imágenes}
\label{ch:det_mal}

%PONER MÁS SECCIONES

\section{¿Qué es la esteganografía?}

La esteganografía es el arte de ocultar información a simple vista dentro, o por encima, de algo que no es secreto. Por ejemplo, un documento de texto, un cuadro, una grabación... Sin embargo, no es nada nuevo.%cita

\subsection{Un poco de historia}

Ya en la Antigua Grecia se utilizó para prevenir la invasión del imperio persa gracias a esta técnica: Demarato, antiguo rey de Esparta, tras ser derrocado por Leotíquidas, encontró refugio en la corte del rey de Persia Darío I. Pasó un tiempo e incluso formó parte en la designación de Jerjes como sucesor de Darío. No obstante, a pesar de haber sido expulsado sentía lealtad por su tierra y, cuando los persas se disponían a invadir Grecia, avisó a los griegos mediante mensajes ocultos usando cera para ello; así, pudieron rearmarse y preparar la invasión. %cita %cita

Más tarde, en los siglos XVI y XVII también se le daría un buen uso a esta técnica con métodos propuestos a partir de nuevas formas de codificación.%cita

Incluso durante la Primera Guerra Mundial se aplicó tinta invisible como aplicación de esta técnica. Durante la Segunda Guerra Mundial también se llegó a utilizar un método muy interesante: los \textbf{micropuntos}. Los micropuntos, básicamente, eran fotografías del tamaño de una página reducidas hasta ocupar un punto de 1 mm de diámetro y, para ocultarlos, se pegaban sobre un punto del texto que los encubría. %cita %cita