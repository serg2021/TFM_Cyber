\chapter{Detección de malware en imágenes}
\label{ch:det_mal}

%PONER MÁS SECCIONES

\section{¿Qué es la esteganografía?}

La esteganografía es el arte de ocultar información a simple vista dentro, o por encima, de algo que no es secreto. Por ejemplo, un documento de texto, un cuadro, una grabación... Sin embargo, no es nada nuevo.%cita

\subsection{Un poco de historia}

Ya en la Antigua Grecia se utilizó para prevenir la invasión del imperio persa gracias a esta técnica: Demarato, antiguo rey de Esparta, tras ser derrocado por Leotíquidas, encontró refugio en la corte del rey de Persia Darío I. Pasó un tiempo e incluso formó parte en la designación de Jerjes como sucesor de Darío. No obstante, a pesar de haber sido expulsado sentía lealtad por su tierra y, cuando los persas se disponían a invadir Grecia, avisó a los griegos mediante mensajes ocultos usando cera para ello; así, pudieron rearmarse y preparar la invasión. %cita %cita

Más tarde, en los siglos XVI y XVII también se le daría un buen uso a esta técnica con métodos propuestos a partir de nuevas formas de codificación.%cita

Incluso durante la Primera Guerra Mundial se utilizó tinta invisible como aplicación de esta técnica. Durante la Segunda Guerra Mundial también se llegó a utilizar un método muy interesante: los \textbf{micropuntos}. Los micropuntos, básicamente, eran fotografías del tamaño de una página reducidas hasta ocupar un punto de 1 mm de diámetro y, para ocultarlos, se pegaban sobre un punto del texto que los encubría. %cita %cita

\section{Esteganografía digital}

Con la modernización de la sociedad en cuanto a tecnología y computación se refiere, aparecieron diversas aplicaciones de esta técnica con las que se consiguió un nivel de sofisticación y detalle en la técnica adecuado a la época: ahora no sólo era mucho más fácil realizar el escondite del mensaje, además era más complicado detectarlo. %cita 

A continuación mencionaremos de forma superficial algunas técnicas  de esteganografía digital para poner en contexto la base sobre la que se va a realizar el trabajo:

\subsection{Clasificación tradicional}
\label{sec:trad}

Este tipo de clasificación se basa en cómo se ocultan los datos. A continuación vamos a ver algunas de las técnicas que se comprenden dentro de esta rama:

\subsubsection{Técnicas basadas en la inserción (Insertion-Based)}

Estas técnicas trabajan con la inserción de bloques de datos dentro de un archivo publicable. El dato es insertado en el mismo punto del archivo. Se suelen ocultar en las partes redundantes de un archivo, de forma que el bloque de datos puede estar dividido en varias secciones, haciendo su detección muy difícil.%cita

Un ejemplo de uso muy bueno para esta técnica es insertar en los \ac{LSB} de un archivo de 8 ó 16 bits. Lo que se hace es codificar el mensaje en binario e introducirlo dentro de los bits 0, 1, 2 y, tal vez, 3. De esta forma, al ser una de las partes más redundantes de información del archivo, el usuario no es consciente de que ha sido modificado, y por lo tanto puede interactuar con el archivo sin saber que puede haber un mensaje o incluso un malware en su interior que se esté ejecutando.

\subsubsection{Técnicas basadas en algoritmos (Algorithmic-Based)} 

Estas técnicas utilizan alguna clase de algoritmo para decidir dónde se debería ocultar un mensaje dentro de un archivo de datos: es por ello que el bloque de datos no se inserta siempre en el mismo punto, puede estar tanto en las partes más redundantes y olvidadas como en las más importantes... Lo cual es un problema, ya que si no se efectúa bien el algoritmo la ocultación de datos sería fácilmente detectable si se comparan el archivo original con el modificado. %cita

El quid de la cuestión de estas técnicas reside en utilizar un algoritmo idóneo para cada tipo de archivo. Optimizando el espacio del archivo con un buen algoritmo, los resultados para la ocultación serán idóneos.

\subsubsection{Técnicas basadas en gramática (Grammar-Based)}

Existe una gran diferencia entre estas técnicas y las anteriores: para las dos primeras se utiliza siempre un archivo que será el que encubra el mensaje... Sin embargo, para estas técnicas se genera un archivo desde cero a partir de los datos a ocultar, basándose en una gramática predefinida. %cita

\subsection{Clasificación moderna}

A diferencia de \ref{sec:trad}, esta clasificación se basa en cómo y dónde se ocultan los datos. Es idónea para representar las técnicas desarrolladas en la esteganografía moderna.

\subsubsection{Técnicas basadas en la inserción (Insertion-Based)}

En esencia, se trata de técnicas cuyo objetivo es ocultar datos insertándolos en un archivo sin afectar a su representación, aunque se aumente su tamaño.

Lo bueno de estas técnicas es que se pueden introducir datos en el archivo sin afectar ni modificar el contenido existente. Lo malo es el incremento de tamaño, lo cual es un claro indicador de que se han introducido datos añadidos al archivo.

\subsubsection{Técnicas basadas en la sustitución (Substitution-Based)}

Estas técnicas se basan en sustituir partes del archivo encubridor por los datos a ocultar. De esta forma el tamaño del archivo no varía y, a priori, podría parecer un archivo normal y corriente.

No obstante, es importante aclarar que la sustitución de datos no puede realizarse en cualquier zona del archivo: se corre el riesgo de que al modificarlo, haya partes que queden inservibles o visualmente defectuosas. La clave es encontrar zonas redundantes del archivo y modificarlas para que no se dé ningún impacto.

Otro factor que hay que tener en cuenta es la cantidad de datos que se pueden ocultar en el archivo hasta que éste finalmente no pueda ejecutarse correctamente. En general es una técnica muy buena siempre y cuando se cumplan estas reglas.

\subsubsection{Técnicas de generación (Generation-Based)}

De la misma forma que hemos visto en las técnicas de \ref{sec:trad}, estas tienen una principal diferencia que cabe destacar: el archivo (o los datos) a ocultar es utilizado para crear el archivo definitivo.

De esta forma, si se consigue únicamente este archivo no debería haber problemas para que el usuario no detecte el mensaje oculto. Si se consigue el archivo original y se compara con el que tiene los datos ocultos se verá que tiene una composición binaria diferente, y la detección esteganográfica será más complicada de realizar.

\section{Stegosploit}

