%*************************************************************************
%	PLANTILLA PARA LA EDICIÓN DE PROYECTOS FIN DE CARRERA
%	Departamento de Teoría de la Señal y Comunicaciones
%	Universidad de Alcalá
%*************************************************************************

%******************
%Tipo de documento
%******************
%Descomentar la opción que corresponda:
%Versión durante la revisión (incluye números de línea)
\documentclass[12pt,twoside]{book}
\usepackage{lineno}
\linenumbers
%Versión final para imprimir a doble cara
%\documentclass[12pt,twoside]{book}
%Versión final para imprimir a una sola cara
%\documentclass[12pt,oneside]{book}

%********************
%Paquetes a utilizar
%********************

\usepackage[square,numbers]{natbib}
\usepackage[T1]{fontenc}
\usepackage[spanish]{babel}
\usepackage{verbatim}
\usepackage{amssymb}
\usepackage{amsmath}
\usepackage{fancyhdr}
\usepackage{graphicx}
\usepackage{multicol}
\usepackage{makeidx}
%\usepackage[bf,SL,BF]{subfigure}
\usepackage{subcaption}
\usepackage{caption}
\usepackage[utf8]{inputenc}
\usepackage{color}
\usepackage{multirow}
\usepackage{float}
\usepackage[printonlyused]{acronym}


\usepackage[pagebackref=true,breaklinks=true,letterpaper=true,colorlinks,bookmarks=false]{hyperref}
%No incluir paquetes depués de este.

%*************
%Definiciones
%*************
\deactivatetilden 
\addto\captionsspanish{\def\tablename{Tabla}\def\listtablename{Lista
de tablas}\def\listfigurename{Lista de figuras}}
%Cambios en los márgenes
\renewcommand{\baselinestretch}{1.1}

\setlength{\oddsidemargin}{0.4cm}
\setlength{\evensidemargin}{-0.4cm} \setlength{\headsep}{0.75cm}
\setlength{\textheight}{23cm} \setlength{\textwidth}{16cm}
\setlength{\topmargin}{-0.25cm}\flushbottom



%***************
%Título del TFC
%***************
\title{Sistema de adaptación inteligente de la velocidad para vehículos basado en IA y visión por computador}

%****************
%Autor
%****************
\author{Sergio Sastre Arrojo}

%Prepara el índice
\makeindex



%***********************
%Comienzo del documento
%***********************
\begin{document}
\setcounter{tocdepth}{4} \setcounter{secnumdepth}{3}

\frontmatter

%Incluimos fichero previo.tex -> genera la hoja de calificación y la portada
\begin{titlepage}
%**********************************************************
%GENERA LA HOJA OFICIAL DE CALIFICACIÓN PARA PROYECTOS FIN DE CARRERA
%SEGÚN EL FORMATO DE LA UNIVERSIDAD DE ALCALÁ Y DEL DEPAR-
%TAMENTO DE TEORÍA DE LA SEÑAL Y COMUNICACIONES.
%**********************************************************

\begin{center}
\LARGE \textsc{Universidad de Alcalá}\\
\vspace{0.5cm}

%Nombre de la escuela-> seleccione el que corresponda.
\textbf{Escuela Politécnica Superior}\\
%\textbf{Escuela Técnica Superior de Ingeniería Informática}\\

%Titulación -> escriba su titulación
Master en Ciberseguridad
\end{center}

\vspace{0.5cm}

\begin{center}
%Para compilar con latex
%\includegraphics[width=4cm]{Figuras/LogoUAH.eps}\\
%Para compilar con pdflatex
%\includegraphics[width=4cm]{Figuras/LogoUAH.png}\\
\end{center}


\begin{center}
\vspace{1cm}

\LARGE Trabajo Fin de Master\\
\textbf{\Huge \textsc{{Presentación de una IA capaz de detectar información oculta en imágenes basada en esteganografía}}}\\
\vspace{0.5cm}
\large Autor: Sergio Sastre Arrojo\\
Director: Miguel Ángel Sicilia Urbán\\
\vspace{0.5cm}
\end{center}

\begin{flushleft}
\textbf{TRIBUNAL:}\\
\vspace{1.5cm}
\textit{Presidente: }\\
\vspace{1.5cm}
\textit{Vocal 1º: }\\
\vspace{1.5cm}
\textit{Vocal 2º: }\\
\vspace{1.5cm}
\textbf{CALIFICACIÓN:}............................................................ FECHA:.................... \\
%IMPORTANTE!
%La copia en PDF para la biblioteca, y uno de los tomos no lleva espacio para la PUBLICACIÓN!
\end{flushleft}
%Final de la hoja de calificación.

%Introduce hoja en blanco
\newpage
\thispagestyle{empty}
\hspace*{0.5cm}
\newpage


\end{titlepage}

%Prepara el título del proyecto.
\maketitle

%Introduce hoja en blanco
\newpage
\thispagestyle{empty}
\hspace*{0.5cm}
\newpage

%Incluye las dedicatorias (fichero dedicatorias.tex)
%**********************************************************
%GENERA LA PORTADA OFICIAL PARA PROYECTOS FIN DE CARRERA
%SEGÚN EL FORMATO DE LA UNIVERSIDAD DE ALCALÁ Y DEL DEPAR-
%TAMENTO DE TEORÍA DE LA SEÑAL Y COMUNICACIONES.
%**********************************************************
\setcounter{page}{0}

\begin{center}
\LARGE \textsc{Universidad de Alcalá}\\
\vspace{0.5cm}

%Nombre de la escuela-> seleccione el que corresponda.
\textbf{Escuela Politécnica Superior}\\
%\textbf{Escuela Técnica Superior de Ingeniería Informática}\\

%Titulación -> escriba su titulación
La titulación del alumno\\
\end{center}

\vspace{0.5cm}

%Logotipo de la universidad

\begin{center}
%Para compilar en latex
%\includegraphics[width=4cm]{Figuras/LogoUAH.eps}
%Para compilar con pdflatex
%\includegraphics[width=4cm]{Figuras/LogoUAH.png}
\end{center}


\begin{center}
\vspace{1cm}

\LARGE Proyecto Fin de Carrera\\

\vspace{0.5cm}

\textbf{\Huge \textsc{{Título del Proyecto fin de carrera}}}\\

\vspace{1.5cm}

\large Autor: Nombre y apellidos\\
Director: Nombre y apellidos\\

\vspace{1.5cm}

\large Año que corresponda 2007,2008, ...\\
\end{center}


%Final de la portada.

%Incluye los agradecimientos (fichero agradecimientos.tex)
\chapter{Agradecimientos}
Poco a poco me estoy convirtiendo en el profesional que quiero llegar a ser, y el camino que estoy recorriendo para ello está lleno de personas que merecen ser agradecidas en este trabajo, ya que sin su apoyo y sus lecciones no podría haber llegado aquí. En primer lugar, quiero agradecer a mi familia su apoyo incondicional durante el curso, a mis amigos su lealtad y honestidad en situaciones complicadas, a mis compañeros y profesores su tiempo y sus lecciones que me acompañarán para toda la vida y, por último; a mis abuelos, de quienes he podido aprender aptitudes como la perseverancia, la resiliencia y el trabajo duro por encima todo.

%Introduce hoja en blanco
\newpage
\thispagestyle{empty}
\hspace*{0.5cm}
\newpage

%Inlcuimos el índice que latex genera automáticamente.
\tableofcontents
%La lista de figuras incluidas en el proyecto.
\listoffigures
%La lista de tablas.
\listoftables


%Incluye el resumen del Proyecto fin de carrera (fichero resumen.tex)
\chapter{Resumen}
%\textcolor{red}{Resumen de 100 palabras MÁXIMO}\\

%En el siguiente trabajo vamos a tratar de solucionar un problema que lleva existiendo desde la invención del automóvil de una forma que no se había visto antes. Con la creación de las normas que regulan el tráfico en las carreteras y la fabricación masiva de vehículos, es un hecho que la probabilidad de que ocurra un accidente es muy alta. Para solucionarlo se inventaron los sistemas \ac{ISA} \cite{reduccion} que ayudaron a reducir este número... Sin embargo, no son perfectos y tienen fallos.

%Es por ello que aquí sugerimos nuestra solución con un nuevo y mejorado sistema: \[ISA^{2}\]
En el siguiente trabajo mostramos una versión mejorada del sistema de adaptación de la velocidad $ISA^{2}$. Éste consiste en la estimación apropiada de la velocidad para un vehículo basándose en la situación del tráfico en un determinado momento capturada por una cámara. A diferencia de otros sistemas, éste predice la velocidad apropiada en lugar de la velocidad real, y para ello usamos diferentes sistemas de regresión de inteligencia artificial. El sistema realiza la regresión tomando como entrada una segmentación semántica de la imagen. En concreto, proponemos utilizar el sistema de segmentación conocido como Swiftnet, que combinado con un sistema de regresión empleando la técnica llamada \ac{SPP}, consigue un error medio para la estimación de la velocidad adecuada de tan solo 8.65 km\textbackslash{h}.
%TODO_DONE: Completa esto y luego lo traduces al Inglés abajo.


\vspace{0.5cm}

\textbf{Palabras clave}: visión por computador, inteligencia artificial, sistema de adaptación de la velocidad inteligente, vehículos inteligentes.
%\textbf{Palabras clave}: cinco palabras como máximo, separadas por comas.

%Introduce hoja en blanco
\newpage
\thispagestyle{empty}
\hspace*{0.5cm}
\newpage

\chapter{Abstract}
%\textcolor{red}{Resumen de 100 palabras MÁXIMO - Es OBLIGATORIO}\\
In this work, we show an improved version of the intelligent speed adaptation system  $ISA^{2}$. It consists in the appropriate speed estimation for a vehicle on the basis of the traffic situation at a certain point of time captured by a camera. Unlike other systems, this one predicts the appropriate speed instead of the real speed, that is why we use different artificial intelligence regression systems. The system performs the regression taking as input the semantic segmentation of the image. To be more specific, we propose to use the segmentation system known as Swiftnet, which is combined with a regression system using the \ac{SPP} technique, getting a mean error for the estimation of the appropriate speed of just 8.65 km\textbackslash{h}.

\vspace{0.5cm}

\textbf{Keywords}: computer vision, artificial intelligence, intelligent speed adaptation, intelligent vehicles.
%\textbf{Keywords}: cinco palabras como máximo, separadas por comas.
%Introduce hoja en blanco
\newpage
\thispagestyle{empty}
\hspace*{0.5cm}
\newpage


%Incluye el resumen Extendido del Proyecto fin de carrera (fichero resumen-extendido.tex)
\chapter{Resumen Extendido}
%Debe incluir un resumen del trabajo de un máximo de 5 páginas. Resaltar aspectos fundamentales del desarrollo, los resultados más relevantes y las conclusiones.

Como objetivo principal, en este trabajo se persigue presentar un modelo de Deep Learning que sea capaz de detectar malware oculto en imágenes por esteganografía. Con este modelo pretendemos facilitar la experiencia del usuario en su navegador de confianza desde un enfoque más seguro, sabiendo que los ciberataques utilizados mediante el envío de imágenes maliciosas es un tema que está a la orden del día.

Actualmente, existe una variedad pasmosa de ciberataques dirigidos tanto a usuarios como a organizaciones. Varios de ellos se basan en el envío de malware y páginas de phishing mediante imágenes y links maliciosos. En este trabajo nos centraremos en buscar una forma de evitar los primeros, usando como móvil las imágenes y como herramienta la esteganografía. Para ello vamos a basarnos en la tecnología \textbf{Deep Learning} como principal solución ante este problema.

De forma resumida, vamos a introducir el malware en un set de imágenes de COCO (\cite{coco}) por esteganografía utilizando la herramienta \textbf{Stegosploit} (\cite{stegosploit}). No se lo introduciremos al set completo, de 5000 imágenes 250 tendrán el malware en su interior; así conseguiremos tener un set de imágenes preparado para poder entrenar una serie de modelos de Deep Learning con los que detectar el exploit de forma óptima. %cita %cita

Como ya hemos dicho, utilizaremos Stegosploit como herramienta principal para introducir el código malicioso en las imágenes. Stegosploit es una herramienta que consiste en introducir en la carga útil del archivo de imagen la información que se quiere ocultar. Una imagen se compone de cientos de píxeles, cada uno de los cuales se puede componer de tres canales de color: \textbf{rojo}, \textbf{verde} y \textbf{azul}; cada canal a su vez se compone de 8 bits, de modo que un píxel se puede componer de 24 bits. La herramienta se servirá de estos bits para modificarlos e introducir el exploit de forma que se codifique como un flujo de bits; es decir, el malware se codificará como una consecución de bits y se insertará en los bits de los píxeles que componen la imagen modificando sus valores según éstos lo requieran.

Para saber en qué bits se ha de insertar el código, conviene representar a la imagen con otro enfoque: ya no es un conjunto de píxeles, vamos a ir más allá, ahora es una consecución de \textbf{``capas de bits''}. Como ya hemos dicho antes los píxeles se componen de canales con 8 bits cada uno, de modo que habrá bits que tengan un mayor peso ,es decir, más importancia en la imagen que otros. Esta diferencia de pesos se puede representar en 8 capas (una capa por bit), siendo las más bajas las que menor peso tienen. Con Stegosploit el objetivo es insertar el código en la imagen en las capas más bajas, ya que son las que menos información aportan y, visualmente, las que menos destacan. Si se introdujese en las capas más altas se vería que la imagen ha sido alterada y por lo tanto su detección sería instantánea. De esta forma, al modificar los valores de los bits más bajos para componer el código que queremos introducir, la detección a simple vista no es posible, ya que con los valores más bajos tan solo se modifica el valor compuesto del píxel de forma muy sutil y con variaciones muy pequeñas.

Para la detección nos basaremos en modelos capaces de realizar una clasificación por imágenes: esto es, los modelos se basarán en dos tipos de etiquetas para las imágenes (\textbf{``imágenes normales''} e \textbf{``imágenes con exploit''}) y conseguirán clasificar las imágenes del set para éstas. Dichas etiquetas las definimos según el nombre del archivo de imagen: las que comiencen con \textbf{mayúscula} serán imágenes normales y las que empiecen por \textbf{minúscula} serán imágenes con exploit. El objetivo de la clasificación no es otro que conseguir que los modelos aprendan, a nivel de bit, qué imágenes pueden contener información oculta y cuáles no. Cuanto más se entrenen con un número mayor de imágenes, mejor; así lograremos que para cualquier imagen (independientemente del nombre del archivo) los modelos sepan cuáles son maliciosas y cuáles no.

Para empezar vamos a trabajar con Stegosploit. Para trabajar con Stegosploit primero debemos crear un servidor HTTP local que introduzca una cabecera especial en cada respuesta porque, como veremos en el capítulo \ref{ch:stego}, si no lo hacemos no se podría ejecutar la herramienta (\cite{server-http}). Creando dicho servidor y añadiendo una etiqueta HTML al código fuente de la herramienta, podemos hacer uso de la misma (como ya hemos dicho, usaremos 250 imágenes para introducir el malware en este paso).

Una vez introducido el exploit en las imágenes pasamos a entrenar una serie de modelos de Deep Learning para realizar la detección, concretamente tres: \textbf{ResNet-34} (\cite{resnet34}), \textbf{ResNet-18} (\cite{resnet18}) y \textbf{Alexnet} (\cite{alexnet}).

Para trabajar con ellos usaremos \textbf{FastAI} (\cite{fastai}), una librería de Python diseñada para entrenar modelos de Deep Learning de forma eficiente, rápida y sencilla; y para trabajar con esta librería, como no tenemos en nuestro haber una \ac{GPU}, utilizaremos los servicios de Google Colab (\cite{google-colab}). Así entrenaremos los modelos.

Para medir la eficacia de los entrenamientos nos serviremos de 4 tipos de métricas: \textbf{Error Rate} (\cite{error-rate}), \textbf{Average Precision Score} (\cite{apscore}), \textbf{Balanced Accuracy} (\cite{balanced-accuracy}, \cite{balanced-accuracy-2}) y \textbf{ROC AUC Score} (\cite{roc-auc-score}).

Según nuestros experimentos, con dos de los tres modelos cotejados se consiguen obtener unos resultados realmente esperanzadores para poder utilizarse en un futuro: \textbf{ResNet-34} y \textbf{ResNet-18}. Sin embargo, aun teniendo unos valores muy parejos entre sí, finalmente nos decantamos por \textbf{ResNet-34} como estandarte para la realización de trabajos de la misma rama con recursos y objetivos más avanzados.

\begin{table}[H]
\centering
\resizebox{12cm}{!}{
\begin{tabular}{|c|c|c|c|}\cline{1-4}
\textbf{Modelos} & \textbf{Métricas} & \textbf{Epoch} & \textbf{Resultados}\\ \cline{1-4}
\multirow{4}{3cm}{\textbf{\textit{ResNet-34}}} & \multirow{2}{3cm}{Error Rate} & 0 & 0.049 \\ \cline{3-4}
& & 24 & 0.002 \\ \cline{2-4}
& \multirow{2}{3cm}{Average Precision Score} & 0 & 0.60 \\ \cline{3-4}
& & 24 & 0.98 \\ \cline{2-4}
& \multirow{2}{3cm}{Balanced Accuracy} & 0 & 0.92\\ \cline{3-4}
& & 24 & 1 \\ \cline{2-4}
& \multirow{2}{3cm}{ROC AUC Score} & 0 & 0.88 \\ \cline{3-4}
& & 24 & 1 \\ \cline{1-4}
\multirow{4}{3cm}{\textbf{\textit{ResNet-18}}} & \multirow{2}{3cm}{Error Rate} & 0 & 0.036 \\ \cline{3-4}
& & 24 & 0 \\ \cline{2-4}
& \multirow{2}{3cm}{Average Precision Score} & 0 & 0.57 \\ \cline{3-4}
& & 24 & 1 \\ \cline{2-4}
& \multirow{2}{3cm}{Balanced Accuracy} & 0 & 0.80\\ \cline{3-4}
& & 24 & 0.99 \\ \cline{2-4}
& \multirow{2}{3cm}{ROC AUC Score} & 0 & 0.89 \\ \cline{3-4}
& & 24 & 0.99 \\ \cline{1-4}
\end{tabular}
}
\caption{Resultados de ResNet-34 y ResNet-18}
\label{tab:Resul_ResNet}
\end{table}

En la tabla \ref{tab:Resul_ResNet} podemos ver la comparativa entre ambos modelos de la familia ResNet. Como se puede apreciar, los dos dan unos resultados muy parecidos entre sí, y en muy pocas métricas se puede ver uno que destaque sobre el otro: en el caso de \textbf{ResNet-18} podemos comprobar que sus valores en \textbf{Error Rate} son mejores que con \textbf{ResNet-34}; y, análogamente, lo mismo pasa con \textbf{ResNet-34} para la métrica de \textbf{Balanced Accuracy}.

ResNet-34 y ResNet-18 acaban finalmente siendo los mejores modelos, pero hemos de decidir cuál de ellos resultará en una mejor utilización para el futuro. Observando los resultados de la tabla, concluimos en que \textbf{ResNet-34} acaba siendo un mejor modelo, aunque no por demasiada diferencia, por encima de \textbf{ResNet-18}. Lo decidimos así en gran parte por la métrica de \textbf{Balanced Accuracy}, que indica de forma aproximada cómo de bien rinde un modelo de Deep Learning; por otro lado, elegimos ResNet-34 por la profundidad de las capas convolucionales que se utilizan en la \ac{CNN} que lo conforma (\cite{cnn}): a mayor profundidad, mayor detalle y precisión en el análisis de imágenes, y por lo tanto mejores resultados en la clasificación.

Sin embargo, y como indicaremos más adelante, con un set de imágenes como el que usamos no es suficiente para entrenar el modelo y que éste sea capaz de detectar el malware de la imagen a nivel de bit: hacen falta más entrenamientos con sets de imágenes más preparados. Esto, como ya hemos dicho, lo indicaremos más adelante.

Por todo ello, y para concluir, se puede decir que: ResNet-34 es el modelo elegido por nosotros como el mejor candidato para ser entrenado por nuevos sets de imágenes y aplicaciones que persigan los objetivos de este trabajo, pero con una mayor incisión en el problema y tratándolo con herramientas y recursos más avanzados. Hemos utilizado \textbf{Stegosploit} y \textbf{FastAI}, pero estas herramientas tan sólo son la punta del iceberg dentro de un mundo que no deja de avanzar.

%Incluimos lista de abreviaturas
\chapter{Glosario}

%Lista con los acrónimos

\begin{acronym}[ISA]
\item \acro{ISA}{Intelligent Speed Adaptation}
\end{acronym}

\begin{acronym}[GPS]
\item \acro{GPS}{Global Positioning System}
\end{acronym}

%\begin{acronym}[PFC]
%\item \acro{PFC}{Proyecto Fin de Carrera}
%\end{acronym}

\begin{acronym}[SS]
%\item \acro{SS}{Proceso por el cual los píxeles de una imagen son dotados de distintos valores para poder diferenciarlos en etiquetas unos de otros y así reconocer los elementos que componen dicha imagen. Por ejemplo: Una fotografía de una persona, un coche y un perro. A priori, todos los píxeles de la imagen no están categorizados y no se sabe qué partes de la imagen corresponden a la persona, al coche, y al perro. Gracias a la segmentación semántica los píxeles de la imagen adquieren los valores de las etiquetas referentes a "persona", "coche" y "perro"; y son fácilmente diferenciables.}
\item \acro{SS}{Segmentación Semántica}
\end{acronym}

\begin{acronym}[mIoU]
\item \acro{mIoU}{Mean Intersection Over Union}
\end{acronym}

\begin{acronym}[IoU]
\item \acro{IoU}{Intersection Over Union}
\end{acronym}

\begin{acronym}[CNN]
%\item \acro{CNN}{Tecnología que posibilita la predicción de etiquetas de una imagen para la segmentación semántica}
\item \acro{CNN}{Convolutional Neural Network}
\end{acronym}

\begin{acronym}[DCNN]
%\item \acro{CNN}{Tecnología que posibilita la predicción de etiquetas de una imagen para la segmentación semántica}
\item \acro{DCNN}{Deep Convolutional Neural Network}
\end{acronym}

\begin{acronym}[SPP]
\item \acro{SPP}{Spatial Pyramid Pooling}
\end{acronym}

\begin{acronym}[ABI]
\item \acro{ABI}{Application Binary Interface}
\end{acronym}

\begin{acronym}[GPU]
\item \acro{GPU}{Graphical Processing Units}
\end{acronym}

\begin{acronym}[CPU]
\item \acro{CPU}{Central Processing Units}
\end{acronym}

\begin{acronym}[IDE]
\item \acro{IDE}{Integrated Development Environment}
\end{acronym}

\begin{acronym}[MAE]
\item \acro{MAE}{Mean Absolute Error}
\end{acronym}

\begin{acronym}[SVR]
\item \acro{SVR}{Support Vector Regression}
\end{acronym}

\begin{acronym}[SVM]
\item \acro{SVM}{Support Vector Machine}
\end{acronym}

\begin{acronym}[FPS]
\item \acro{FPS}{Frames Per Second}
\end{acronym}
%Para poner más acrónimos
%\acrodef{CNN}{Convolutional Neural Network}
%\acrodef{SS}{Segmentación Semántica}







\mainmatter

%Incluimos los capítulos del proyecto. Cada capitulo se encuentra en un fichero tex,
%que insertamos en el documento por medio del comando \include{capitulo-cualquiera}

%Incluye Introducción (fichero introduccion.tex)
%**********************************************************
%GENERA LA PORTADA OFICIAL PARA PROYECTOS FIN DE CARRERA
%SEGÚN EL FORMATO DE LA UNIVERSIDAD DE ALCALÁ Y DEL DEPAR-
%TAMENTO DE TEORÍA DE LA SEÑAL Y COMUNICACIONES.
%**********************************************************
\setcounter{page}{0}

\begin{center}
\LARGE \textsc{Universidad de Alcalá}\\
\vspace{0.5cm}

%Nombre de la escuela-> seleccione el que corresponda.
\textbf{Escuela Politécnica Superior}\\
%\textbf{Escuela Técnica Superior de Ingeniería Informática}\\

%Titulación -> escriba su titulación
La titulación del alumno\\
\end{center}

\vspace{0.5cm}

%Logotipo de la universidad

\begin{center}
%Para compilar en latex
%\includegraphics[width=4cm]{Figuras/LogoUAH.eps}
%Para compilar con pdflatex
%\includegraphics[width=4cm]{Figuras/LogoUAH.png}
\end{center}


\begin{center}
\vspace{1cm}

\LARGE Proyecto Fin de Carrera\\

\vspace{0.5cm}

\textbf{\Huge \textsc{{Título del Proyecto fin de carrera}}}\\

\vspace{1.5cm}

\large Autor: Nombre y apellidos\\
Director: Nombre y apellidos\\

\vspace{1.5cm}

\large Año que corresponda 2007,2008, ...\\
\end{center}


%Final de la portada.

\include{Detección_de_malware_en_imágenes}
%Incluye un capítulo cualquiera (capitulo-cualquiera.tex)
\chapter{Estado del arte}
\label{ch:sota}

En el siguiente capítulo pasaremos a mencionar los artículos que conforman el estado del arte del proyecto para poder hacer una comparativa tanto de los modelos como de las métricas utilizadas y tener una base sobre la que movernos durante la implementación del mismo, la cual veremos más adelante.

\section{Uso de Machine Learning para detectar imágenes con políglotas maliciosos}

%IMAGEN

El objetivo de este proyecto no era otro que hacer una herramienta de Machine Learning para detectar políglotas en imágenes (\cite{ml-stenography-shawat}). A continuación, pasaremos a listar y explicar de forma resumida el proceso que realizaron los autores para ello. %cita

Este estudio ha sido realizado tomando como base el artículo de Stegosploit del que también nos hemos servido para este trabajo (\cite{stegosploit}). %cita

\subsection{Estudiar y construir ataques XSS y CSRF}

En esta fase se procede a ver cómo se puede distribuir a las víctimas las imágenes con el código malicioso en su interior. Para ello, se estudiaron diferentes tipos de ataques sobre los que poder hacerlo, de tal forma que la víctima no se diera cuenta. A continuación, vamos a ver de manera muy superficial en qué consisten:

\subsubsection{Reflected XSS}

Este ataque se basa en la posibilidad de que un atacante modifique una URL de una aplicación con código malicioso. Si esta URL modificada es enviada a un usuario porque éste, por ejemplo, la solicita; el código malicioso se ejecutará en el navegador de la víctima, dentro de la sesión que tenga abierta en la aplicación (\cite{reflected-xss}). %cita

\subsubsection{DOM XSS}

Este ataque, no dista demasiado del anterior, sin embargo tiene otro enfoque: El atacante crea un payload, por ejemplo en la URL de la aplicación, que se ejecuta en el navegador de la víctima debido a una modificación del entorno del DOM usado en el código original de la misma. Así el cliente ejecuta el código del atacante de manera inesperada. De este modo, la página web como tal no cambia, pero en la parte de la víctima, el código contenido en la página se ejecuta de forma diferente por los cambios efectuados en el DOM (\cite{dom-xss}).

Dicho de otra forma, si por ejemplo se crea un payload en la URL y ésta se envía a la víctima, se modifica el entorno del DOM, el cual detecta el payload en la URL y se ejecuta como un objeto más del código de la aplicación.

Como se puede ver, el resultado es el mismo que en el ataque anterior, no obstante recorre otro camino para conseguirlo. %cita

\subsubsection{Persistent XSS}

Para realizar este ataque hay que inyectar código directamente en un servidor objetivo. De esta forma, cada vez que el usuario pida la información de dicho servidor, obtendrá el código introducido por el atacante. Los servidores objetivo para este tipo de ataques suelen ser bases de datos, un foro de mensajes, un registro de visitas, etc (\cite{persistent-xss}). %cita

\subsubsection{CSRF}

Este tipo de ataques fuerzan a un usuario a ejecutar acciones no deseadas en una aplicación web en la que ya están autenticados. Mediante un email o un chat, un atacante puede enviar a la víctima un enlace para engañarla y ejecutar acciones de la elección del atacante (\cite{csrf}). %cita

\subsection{Creación de políglotas}

Para crear los políglotas con los que explotar el ataque nos basaremos completamente en Stegosploit (\cite{stegosploit}).

Todo comienza cifrando el código malicioso en los bits de una imagen. Después pasamos a insertar un decodificador en la cabecera de la imagen para crear IMAJS, formando así un políglota. Por último, nos servimos de alguno de los ataques de la fase anterior para atacar a un servidor web con el políglota como recurso.

\subsection{Uso de Machine Learning para detectar imágenes con políglotas maliciosos}

A continuación, pasamos a explicar la última parte de este modelo.

Comenzamos extrayendo 20.000 imágenes del dataset \textbf{CIFAR-10} (\cite{cifar10}) para después insertar en la mitad de ellas código malicioso mediante Stegosploit (\cite{stegosploit}). %cita #cita

Ahora se utiliza el método de árboles de decisión para el dataset completo (80\% de train y 20\% de test). Los árboles de decisión son un método de Machine Learning usado en problemas de regresión y clasificación; el objetivo es crear un modelo que prediga el valor de una variable aprendiendo mediante simples reglas de decisión inferidas por los datos de entrada (\cite{decision-trees}). %cita

Los resultados finales vienen representados según una gráfica en la que se utiliza la siguiente métrica: La velocidad de detección de código por densidad de cifrado y profundidad del árbol.

Con esta métrica se quiere comprobar a qué velocidad detecta este método el código cifrado en las imágenes dependiendo de la densidad del cifrado y de la profundidad del árbol; es decir, dependiendo de en cuántos bytes se han modificado los bits y con qué distancia están separados éstos entre sí, y dependiendo de la cantidad de ramificaciones del árbol de decisión según las reglas que ha ido adoptando.

\section{Stegomalware}

Uno de los trabajos más recientes en cuanto a detección de malware en imágenes se refiere, nos presenta una revisión de los modelos de de detección usados durante los últimos tiempos junto con una propuesta de modelo original generada con la base de los mismos (\cite{stegomalware}). %cita

Dentro del trabajo se extiende una alta gama de contenidos como las herramientas usadas para insertar malware en imágenes mediante esteganografía, el tipo de formato de archivos usados para ello, los algoritmos de esteganografía en imágenes... Para este apartado de la memoria sólo trataremos las técnicas de estegoanálisis usadas para la detección de esteganografía en imágenes, el modelo de detección propuesto por los autores y las métricas usadas durante los experimentos.

\subsection{Técnicas de estegoanálisis}

El estegoanálisis es el acto de determinar si hay datos ocultos en el medio en el que supuestamente están, y el algoritmo esteganográfico usado para introducirlo (además de extraer los datos si los hubiere). A continuación vamos a pasar a explicar cronológicamente, y de forma resumida, el estado del arte de estas técnicas:

\subsubsection{Soluciones de modelos de estegoanálisis enriquecidos y de características basadas en el dominio de imágenes}

Durante las dos últimas décadas se han ido desarrollando diferentes soluciones para abordar el problema de la esteganografía desde este enfoque. Desde un principio la investigación se concentró en detecciones basadas en Machine Learning y en características de la imagen, seguidas por modelos enriquecidos y por soluciones basadas en clasificadores de conjunto. 

Algunos de los métodos de estegoanálisis como \textbf{PHARM} (\cite{pharm}) y \textbf{GFR} (\cite{gfr}) siguieron siendo efectivos antes de la aparición del Deep Learning. Sin embargo, en el momento en el que surge el Deep Learning, la investigación de estegoanálisis en imágenes convencional se reduce, y es más difícil hacer contribuciones en este ámbito mejorando los métodos anteriormente mencionados. Con el Deep Learning se da un salto de calidad, y la rapidez con la que se desarrolla es mucho más alta en comparación (incluyendo el estegoanálisis). %cita #cita

Aun con la aparición de esta nueva tecnología, el rendimiento ofrecido por estos métodos sirvió como pie de apoyo para mejorar las soluciones que posteriormente aparecerían gracias al Deep Learning. Las soluciones de este apartado se evaluaron y compararon con otras usando métricas de error y de precisión de detección. Los investigadores, además, usaron algoritmos de esteganografía estándar como \textbf{HUGO} (\cite{hugo}), \textbf{UNIWARD} (\cite{uniward}) o \textbf{WOW} (\cite{wow}) para evaluar la efectividad de las soluciones de estegoanálisis.%#cita #cita #cita

Los resultados de la evaluación iban mejorando cada año y, consecuentemente, iban apareciendo más soluciones que probar. Hasta 2015, año de aparición de Deep Learning, técnicas como GRF o PHARM fueron de lo mejor que hubo. No obstante, ninguno de estos modelos fue evaluado según la detección de malware. Es en este punto donde el Deep Learning toma partida con el estegoanálisis.

\subsubsection{Modelos de Deep Learning para el estegoanálisis de imágenes}

Dentro de la lectura de este trabajo, se proponen una buena cantidad de soluciones de Deep Learning para conseguir la tan ansiada detección de esteganografía en imágenes con unos resultados más prometedores que en el enfoque anterior. Las soluciones principalmente se enfocan en proponer cambios en la capa de preprocesado, en la función de activación, en la disposición de las capas convolucionales y en la fusión de bloques, en la clasificación lineal para capturar elementos esteganográficos.

Al experimentar con las soluciones se ha sacado en claro que hay varios modelos como \textbf{SR-Net} o \textbf{Zhu-Net}, que han conseguido muy buenos resultados en lo que a errores y precisión de detección se refiere. De hecho, estos modelos pueden llegar a ser una referencia para hacer comparativas en el futuro cuando surjan nuevas contribuciones.

Sin embargo, y a pesar de que el campo de Deep Learning ofrece muchas más posibilidades y caminos que tomar para seguir desarrollándose, todavía no hay trabajos que utilicen esta tecnología para la detección de malware... A continuación veremos la propuesta del modelo de los autores para abarcar un nuevo enfoque en la tecnología de Deep Learning.

\subsection{Modelo de detección de stegomalware}

El modelo está pensado para aplicarse dentro del ámbito de la empresa y, dependiendo del tamaño de ésta, de la frecuencia de imágenes con esteganografía cotejadas por la misma, el número de archivos multimedia recibidos por la organización y de las operaciones de negocio de ésta; la estructura del modelo variará.

En general, el proceso es como sigue en la imagen:

%PONER IMAGEN

\begin{itemize}
\item Un empleado informa de un archivo de imagen sospechoso en un correo de phishing (por ejemplo), y la imagen será subida a una ubicación de almacenamiento estándar, la cual está aislada del resto de la infraestructura de la aplicación por motivos obvios. Adicionalmente, el equipo de seguridad podrá subir, si precisa, el archivo para analizarlo.
\item El archivo puede ser escaneado por poder tener un posible malware y, además, se puede determinar cuán malicioso es mediante herramientas de detección de firma (\cite{signature-based-detection}). Estas herramientas indican si la amenaza es conocida o no (si no es conocida, se le pone un identificador para el futuro). %cita
\item Si efectivamente el archivo tiene un malware, marcamos la imagen y realizamos acciones preventivas como aislar la máquina infectada o actualizar el hash de la firma del malware de la imagen en las políticas de seguridad de la organización para bloquear el malware de la imagen y detener la infección de la red.
\item Si por el contrario, no se identifica el archivo como malicioso dentro de este sistema de detección preliminar lo pasamos al siguiente paso: Identificar el tipo de archivo que es y el formato que tiene.
\item Actualmente existen muchos tipos de formato en los archivos multimedia, y uno de los principales retos es identificar el malware en estas condiciones en donde el archivo puede tener cualquier tipo de formato.
\item Si el archivo no es multimedia se seguirá el procedimiento normal de enviarlo a la herramienta anti-malware para su análisis.
\item Si por el contrario, sí se identifica como archivo multimedia, debemos realizar un análisis estructural y estadístico del archivo para la detección del malware. El análisis estructural incluye cambios en la marca de tiempo y en las fechas, propiedades del archivo inusuales como el tamaño del archivo, el checksum y la modificación del contenido, anomalías en el contenido de la cabecera \textbf{Exif} (\cite{exif-header})... Para ello, usamos las herramientas de código abierto \textbf{StegSpy} (\cite{stegspy}) y \textbf{stego-toolkit} (\cite{stego-toolkit}). %cita #cita #cita
\item Si hubiera alguna propiedad anómala detectada, se marcaría el archivo como sospechoso para realizar más análisis.
\item Por otro lado, tenemos el análisis estadístico del archivo. Este es realizado para encontrar más pruebas de cuánto de malicioso es el archivo. Las propiedades estadísticas pueden incluir histogramas de bytes y n-gramas del archivo, los cambios de patrón en los píxeles de la imagen o en los frames de un vídeo y cambios en los bit menos significativos de las imágenes. Las herramientas usadas en este tipo de análisis para esta propuesta son: \textbf{StegExpose} (\cite{stegexpose}) y \textbf{Stegdetect} (\cite{stegdetect}). %cita #cita
\item La respuesta colectiva de los resultados de ambos análisis se combinan y se evalúa de nuevo cuán malicioso es el archivo, o si es sospechoso de tener un malware en su interior.
\item Si el archivo se indica como sospechoso o malicioso, se envía a un entorno de \textbf{Cuckoo Sandbox} (\cite{cuckoo-sandbox}) para aplicar un análisis de malware dinámico e identificar las características de comportamiento del archivo.%#cita
\item Hay una alta probabilidad de que el contenido malicioso oculto en el archivo pueda ser extraído y ejecutado como resultado de unas instrucciones de código embebido.Por ejemplo, un código de shell embebido en el archivo de imagen puede ser ejecutado e intentar conectarse a un servidor remoto para ejecutar comandos maliciosos y exfiltrar datos de la organización.
\item De este modo, basándonos en el comportamiento del malware, tenemos que tomar acciones preventivas en el entorno si el archivo acaba siendo finalmente malicioso.
\item Las acciones preventivas pasan por poder actualizar, de nuevo, la firma del malware para los indicadores de compromiso como direcciones IP, dominios, y otras firmas de código hexadecimal para la detección de malware en el entorno de la infraestructura.
\item Sin embargo, si el archivo se comporta con normalidad durante el análisis dinámico, podemos ignorar el archivo para futuras acciones y podemos seguir la pista del mismo por si se diese el caso de que fuese un falso positivo en el futuro.
\end{itemize}

A continuación, pasaremos a revisar las métricas usadas.

\subsection{Métricas}

En esta revisión se han utilizado numerosas métricas para medir las técnicas de esteganografía en imágenes bajo un baremo apropiado para éstas. De forma resumida, vamos a pasar a explicarlas para tomarlas como referencia en nuestro trabajo, a pesar de que nuestro objetivo es diferente al de esta revisión.

\subsubsection{PSNR}

Esta métrica es muy útil para evaluar la calidad de la imagen. Se puede usar para medir la distorsión entre imágenes con esteganografía e imágenes sin ésta. Se sirve de la función \ac{MSE} (o Error Cuadrático Medio), tal y como viene en la siguiente figura (\cite{mse-ssim}): %cita

%IMAGEN

%IMAGEN


Donde la \textit{N} es la máxima diferencia entre píxeles en una imagen. Si el \ac{MSE} es alto en un algoritmo de esteganografía, es más complicado realizar el estegoanálisis que en un algoritmo con el \ac{MSE} bajo. Con la métrica \ac{PSNR} pasa lo mismo: si tiene un valor alto, es más complejo saber si una imagen tiene o no, datos introducidos por esteganografía. Dicho de otra forma, si el \ac{PSNR} está por encima de los 30 dB, es muy difícil para el ojo humano saber distinguir entre una imagen con esteganografía y otra sin ella.

\subsubsection{SSIM}

La métrica \ac{SSIM} se utiliza para la medida de la calidad de imagen. La fórmula toma como base dos imágenes \textit{A} y \textit{B}, junto con sus respectivos valores medios \textit{X} y \textit{Y}; sus varianzas $X^{2}$ y $Y^{2}$; y su covarianza $XY^{2}$:

%IMAGEN

En general, los valores de las variables \textit{K1} y \textit{K2} serán de 0.01 y 0.03 respectivamente. Los resultados de \ac{SSIM} pueden variar desde -1 a 1, siendo el primero un indicativo de que las imágenes con esteganografía son más difíciles de identificar que las imágenes sin ésta. Para un buen algoritmo esteganográfico, lo mejor sería tener un valor bajo en esta métrica.

\subsubsection{EC}

La \ac{EC} es la relación del número total de bits embebidos en una imagen con el tamaño total de la imagen. También se la conoce como \ac{BPP}. Para una imagen con esteganografía con altura \textit{H}, anchura \textit{W}, y siendo el número de bits embebidos \textbf{E}, la \ac{EC} se representa como:

%IMAGEN

\subsubsection{BPNZAC}

El \ac{BPNZAC} es el número de bits embebidos en los coeficientes \ac{DCT} de la imagen (\cite{dct}). Este número se selecciona para escoger la proporción de los bits embebidos, y usado para evaluar el rendimiento. Dicho de otra forma, esta métrica es el equivalente a \ac{EC} en imágenes con formato JPEG. %cita

\subsubsection{QF}

El \ac{QF} es la métrica usada para medir la calidad de una imagen tras la compresión JPEG. Suele usarse tanto en el ámbito de la esteganografía en imágenes como en el estegoanálisis de imágenes JPEG. Se representa en forma de porcentaje y suele rondar entre el 75\% y el 100\%.

\subsubsection{PE}

La \ac{PE}, también conocida como \ac{DER}, de un método de estegoanálisis en imágenes es el número total promedio de imágenes con y sin esteganografía mal identificadas. Es decir, es el número promedio de falsos positivos y falsos negativos. Se representa por la siguiente fórmula:

%IMAGEN

\textit{PFA} es la probabilidad de falsa alarma, la cual indica la probabilidad de que las imágenes normales se clasifiquen como imágenes con esteganografía; y \textit{PMD} es la probabilidad de detecciones fallidas, la cual indica la probabilidad de que se hayan clasificado erróneamente las imágenes con esteganografía como imágenes normales. Esta métrica es usada principalmente para evaluar en sistemas de Machine Learning (y Deep Learning), técnicas de estegoanálisis en imágenes para detectar información oculta.

\subsubsection{DA}

La \ac{DA} es otra métrica usada comúnmente en este ámbito de evaluación de rendimiento en estegoanálisis y esteganografía, ya que mide la relación entre el número total de clasificaciones correctas entre imágenes con y sin esteganografía dividida por el número total de clasificaciones correctas e incorrectas para ambos tipos de imágenes. Esta fórmula se representa de la siguiente forma:

%IMAGEN

Donde \textit{TP} son el número de clasificaciones correctas para imágenes con esteganografía, \textit{TN} son el número de clasificaciones correctas para imágenes normales, \textit{FP} son el número de falsos positivos (clasificaciones incorrectas de las imágenes normales) y \textit{FN} son el número de falsos negativos (clasificaciones incorrectas de las imágenes con esteganografía).

Alternativamente, podemos representar con la siguiente ecuación la \ac{PE}, ya que la suma de ambas métricas siempre da 1:

%IMAGEN

\subsubsection{BER}

La \ac{BER} cuantifica la robustez de los datos embebidos en la imagen. Sea un algoritmo de esteganografía en el que \textit{B} es el número de bits embebidos en la imagen y \textit{BE} es el número de errores ocurridos durante la extracción de los datos embebidos, la ac{BER} se representa como:

%IMAGEN

\subsubsection{MAE}

El \ac{MAE} se identifica como el promedio del valor absoluto de los errores. El error absoluto es el valor absoluto de la diferencia entre los valores predichos y los valores reales. Se representa mediante la siguiente figura:

%IMAGEN

Siendo \textit{A} y \textit{B} dos imágenes monocromáticas con una anchura \textit{W} y una altura \textit{H}.

El \ac{MAE} puede ser usado para medir la calidad de una imagen con esteganografía comparada con una imagen normal.

\subsubsection{QI}

El \ac{QI} es la medida de la distorsión de la imagen usando factores como la pérdida de correlación, la distorsión de luminancia y la distorsión de contraste. Se puede representar por la siguiente fórmula:

%IMAGEN

La \textit{x} y \textit{y} son las imágenes normales y con esteganografía respectivamente, y la media y la varianza de los valores de los píxeles de éstas se representan como \textit{$\bar{x}$}, \textit{$x^{2}$} y \textit{$\bar{y}$}, \textit{$y^{2}$} (\cite{qi}).%#cita

\subsubsection{Conclusión}

Dentro de todas las posibilidades vistas,y de forma convencional, las métricas usadas en la evaluación de rendimiento para estegoanálisis en imágenes suelen ser \ac{DER} y \ac{DA}. Las tendremos en cuenta cuando realicemos los experimentos.

\section{MalJPEG}

En este proyecto se trata una solución basada en Machine Learning para detectar imágenes JPEG maliciosas. Al igual que con Stegomalware primero veremos cómo es la solución propuesta por los autores y bajo qué condiciones puede actuar, seguido de las métricas utilizadas para sus experimentos.

\subsection{Contexto}

Durante mucho tiempo, los archivos con formato JPEG han sido unos de los más utilizados tanto en el ámbito empresarial como en el del individuo. En gran parte ha sido gracias a su compresión con pérdidas basada en \ac{DCT}, y es por ello que se han convertido rápidamente en un atractivo vector de ataque para los cibercriminales. Para ello, la solución que se propone en este artículo se denomina MalJPEG, y su objetivo no es otro que saber clasificar correctamente si una imagen JPEG es maliciosa o no, mediante tecnología basada en Machine Learning.

\subsection{Solución}

El funcionamiento de MalJPEG se basa en extraer estadísticamente, es decir, sin mirar la imagen como tal, características del archivo (por ejemplo, el tamaño del archivo en bytes). Después se aplican diferentes algoritmos de Machine Learning que tomarán como entrada dichos datos, y clasificarán las imágenes como maliciosas o benignas; tal y como viene en la imagen:

%IMAGEN

\subsubsection{Métodos genéricos de extracción de características}

Generalmente existen dos métodos de extracción de características que, en este artículo, se utilizan: los basados en \textbf{histogramas} y la técnica \textbf{Min-Hash}. A continuación, pasamos a explicarlos brevemente:

\begin{itemize}
\item \textbf{Histogramas}
\begin{itemize}
\item Los métodos de extracción basados en histogramas crean un histograma en el que se recogen datos y valores propios del archivo en cuestión, para poder utilizarlos como entradas en algoritmos de Machine Learning. En este artículo, se han usado dos tipos de histogramas: \textbf{Histogramas simples} e \textbf{histogramas de entropía de bytes avanzados}.
\item Para los primeros se usan 2 configuraciones: histogramas basados en valores de bytes e histogramas basados en valores de caracteres. El histograma cuenta la frecuencia de estos valores en el archivo para cada configuración. Para poder comparar archivos con diferente tamaño es necesario normalizar los valores de los histogramas entre 0 y 1.
\item Para los segundos, se desliza una ventana de bytes de tamaño \textit{K}, con un salto de \textit{S} bytes, sobre la representación en bytes del archivo. Para cada ventana, calculamos la entropía en base 2 del archivo (véase la \textbf{figura fórmula entropía}). Guardamos cada valor individual de los bytes en la ventana con el valor de la entropía de toda la ventana (en pares) en una lista; de este modo, habrá \textit{K} pares por cada ventana. Por último, calculamos un histograma en 2 dimensiones con un eje \textit{E} que representará la entropía, y un eje \textit{B} que representarán los bytes. En este punto es donde se obtiene el vector de carcterísticas de la imagen basado en el histograma.

%IMAGEN
\end{itemize}
\item \textbf{Min-Hash}
\begin{itemize}
\item Esta técnica se basa en estimar rápidamente cómo de similares son dos objetos. El método Min-Hash genera una firma de tamaño \textit{N} para un archivo dado basada en \textit{N} simples funciones hash (\cite{min-hash}). Después se calcula la distancia de Hamming entre las firmas de los dos objetos a comparar: cuanto más pequeña la distancia, más coincidencias en las firmas (\cite{hamming}). %cita
\item La firma de Min-Hash se basa en \textit{shingles} extraídas del archivo. Un shingle es una secuencia de longitud fija con unidades básicas del archivo (bytes, caracteres...). Cada shingle se extrae mediante una ventana de tamaño \textit{W} con un salto de \textit{S} bytes por todo el archivo (como las ventanas de tamaño \textit{K} en los histogramas basados en entropía de bytes). Se aplican \textit{N} funciones hash en cada shingle extraído, y los resultados de los hashes se guardan en un vector de tamaño \textit{N}.
\item Finalmente, se consigue la firma, la cual es un vector de tamaño \textit{N} que contiene el mínimo valor de hash producido por cada función hash, a través de todos los shingles.
\end{itemize}
\end{itemize}

\subsubsection{Algoritmos de Machine Learning}

De cara a utilizar las características de la imagen como parámetros de entrada de la parte de Machine Learning de esta solución, los autores cotejaron varios tipos de algoritmos usados comúnmente en el ámbito de clasificación de imágenes. Dentro de estos tenemos:

\begin{itemize}
\item \textbf{Decision Tree}
\item \textbf{Random Forest}
\item \textbf{Gradient Boosting en Decision Tree (XGBoost y LightGBM)}
\end{itemize}

Por lo general, estos algoritmos de clasificación son muy apropiados para datasets con más cantidad de objetos de una clase que de otra. Por otro lado, en este artículo también el clasificador \textbf{K-Nearest Neighbors} en los datasets de Min-Hash; ya que éste es el único capaz de comparar firmas de Min-Hash usando la función de la distancia de Hamming.

\subsubsection{Métricas}

Para evaluar la clasificación de las imágenes se han cogido métricas como la \ac{TPR} y la \ac{FPR} (derivadas presumiblemente de la \ac{DER}). Por otro lado, y considerando estas métricas podemos utilizar la \ac{ROC}: la \ac{ROC} es la curva creada en una gráfica entre los valores de la \ac{TPR} y la \ac{FPR} en diferentes umbrales. Teniendo la \ac{ROC} podemos medir la \ac{AUC}, es decir, el área bajo la curva de la \ac{ROC}; de este modo si el valor de la \ac{AUC} es alto, quiere decir que el valor de la \ac{TPR} también lo es, mientras que el valor de la \ac{FPR} es bajo (\cite{auc}). %cita

Sin embargo, no sólo se necesitan esas métricas. Hace falta una que optimice los resultados de la detección de los clasificadores, sirviéndose de la \ac{TPR} y la \ac{FPR}: la \ac{IDR} (\cite{idr}). %cita

%IMAGEN

La \ac{IDR} indica el punto del clasificador en el que se ha optimizado adecuadamente, y todo pasa por tener un valor muy alto en \ac{TPR} y uno muy bajo en \ac{FPR}. Cuanto mayor sea la \ac{IDR}, tendremos un mejor clasificador.

\section{ImageDetox}

Esta solución se creó con el objetivo de detectar y neutralizar código malicioso oculto en imágenes, aunque para ello no utilice tecnología de Machine Learning (\cite{imagedetox}). Sin embargo, es interesante comprobar en qué consiste y qué métricas decidieron utilizar los autores. %cita

\subsection{Solución}

ImageDetox es un método creado con el pretexto anterior incluso sin una información previa del código que se ha introducido. El método se compone de cuatro módulos:

\begin{itemize}
\item \textbf{Extracción del archivo de imagen}
\item \textbf{Análisis del formato del archivo de imagen}
\item \textbf{Conversión del archivo de imagen}
\item \textbf{Gestión del archivo de imagen tras la conversión}
\end{itemize}

La solución propuesta puede servir, además, para evitar amenazas de seguridad resultantes del escondite de información confidencial en archivos de imagen, con el objetivo de sacar a la luz dichas amenazas. El esquema de la propuesta es como se muestra en la figura:

%IMAGEN

De forma resumida, ImageDetox funciona de la siguiente manera:
\begin{enumerate}
\item  El primer módulo extrae los archivos de imagen de todos los archivos introducidos al área de Internet (la red local).
\item El segundo módulo identifica el formato de la imagen de la cabecera del archivo extraído en el paso anterior.
\item El tercer módulo aplica una conversión en la imagen con la que se consigue eliminar el código oculto (aun sin saber si tiene uno o no).
\item El cuarto módulo guarda el valor hash de la imagen original en la imagen convertida una vez que se almacena en la unidad de gestión por un período de tiempo. La información del valor de los hashes de las imágenes que han sobrepasado dicho período se borran del almacenamiento, y la información se va  actualizando periódicamente.
\end{enumerate}

\subsection{Métricas}

En este artículo no se utiliza ninguna de las métricas usadas en los otros modelos vistos, aquí se crean dos tipos de métricas únicas para esta solución: \textbf{VT Detection (A)} y \textbf{VT Detection (B)}.

VT Detection (A) hace referencia al número de antivirus que son capaces de detectar malware oculto en una imagen dividido entre el número de antivirus usados para detectar virus en una imagen. Dicho de otra forma, representa los resultados de analizar si cada imagen maliciosa original es maliciosa, según determina VirusTotal (\cite{virustotal}). %cite

Por el contrario, VT Detection (B) hace referencia al resultado de la detección después de utilizar esta técnica de neutralización propuesta en las imágenes con código malicioso. Dicho de otra forma, representa los resultados de analizar si cada imagen maliciosa original es maliciosa tras aplicar esta técnica.

\include{Implementación}

\chapter{Resultados}
\label{ch:res}

En este capítulo pasaremos a explicar cada experimento realizado por cada modelo, las métricas que se usarán para medir los resultados y comentaremos qué modelo consigue mejores prestaciones en base a éstos.

\section{Métricas}

A continuación vamos a explicar en qué consisten las métricas en las que nos basaremos para cotejar los experimentos realizados.

\subsection{Error Rate}

Para saber con cuánta veracidad y exactitud se han clasificado las imágenes usaremos esta métrica denominada \textbf{``Error Rate''} (o \textbf{Tasa de errores}) (\cite{error-rate}). Simplemente indica el número de imágenes bien clasificadas junto con el número de imágenes mal clasificadas; a continuación se calcula en base a estos datos qué porcentaje de error ha habido durante la clasificación.

Realmente es una métrica muy común usada en una gran cantidad de ramas científicas, de modo que no es de sorprender que la hayamos escogido para este proyecto.

\subsection{Average Precision Score}

Con esta métrica se pretende evaluar la precisión media de los resultados de las predicciones. Para ello, se realiza la llamada curva \ac{PR} (o curva de \textbf{Precisión-Sensibilidad}) (\cite{curvas-pr}); con ella se puede ver si la clasificación de las imágenes se ha realizado correctamente, es decir, sin falsos positivos en la misma (\textbf{Precisión}), y si se han detectado bien las imágenes por el modelo dentro de todo el set de imágenes , es decir, si han habido falsos negativos durante el proceso y si dentro de los mismos (y de los positivos reales) se han detectado correctamente los positivos (\textbf{Sensibilidad}). %cita

Es un poco confuso de entender, pero simplemente indica la efectividad del sistema en términos de falsos positivos y falsos negativos. En la curva los dos ejes están muy relacionados entre sí: si se aumenta la precisión, disminuirá la sensibilidad (y al revés).

Sin embargo, esta curva no es la métrica que aquí explicamos. La precisión media se calcula a partir de la misma y es, a grandes rasgos, una forma de representar todos los valores de la curva en uno solo.

Para decirlo de forma resumida, la \textbf{precisión media} toma la curva \ac{PR} y realiza la media ponderada de los valores de la ``Precisión'' adquiridas en cada umbral de decisión, es decir, en cada valor de decisión que define la precisión del sistema (si variamos el umbral, variará a su vez la precisión del sistema, y por lo tanto servirá para dictaminar si el modelo es preciso o no); como pesos para realizar la media ponderada se utilizará el incremento de la sensibilidad del umbral anterior al actual (\cite{apscore}). A continuación mostramos la fórmula para comprender mejor, de forma visual, lo explicado en este punto: %cita

\begin{equation}\label{eq:apscore}
AP = \sum_{n}(R_n - R_{n-1})P_n
\end{equation}

donde $P_n$ es la \textbf{Precisión} y $R_n$ es la \textbf{Sensibilidad} en el umbral \textbf{n}. Cuanto más cercana a 1 esta métrica, mejor será el sistema.

Esta métrica realmente es muy buena para nuestros objetivos porque está pensada para usarse en un set de imágenes \textbf{no balanceado}, es decir, con más elementos del set con una etiqueta que con otra (recordemos que nuestro set se compone de 5000 imágenes, de las cuales 250 tienen un exploit en su interior).

\subsection{Balanced Accuracy}

\textbf{Balanced Accuracy} es una métrica diseñada precisamente para problemas de clasificación con set de imágenes no balanceados, tanto para problemas de clasificación binarios (con dos etiquetas) como de multiclase (con más de dos etiquetas).

Esta métrica se define como la media de la \textbf{Sensibilidad} obtenida de cada clase (recordemos que la \textbf{Sensibilidad} sirve para medir cuántos positivos han sido detectados por el modelo correctamente). Se utiliza para evaluar el rendimiento de un modelo de clasificación. Cuanto más cercano a 1 su valor, mejor será el modelo (\cite{balanced-accuracy}, \cite{balanced-accuracy-2}). %cita

\subsection{ROC AUC Score}

Como ya vimos en el capítulo \ref{ch:sota}, la curva \ac{ROC} se crea a partir de los valores de la \ac{TPR} y la \ac{FPR} en distintos umbrales. Consecuentemente, la \ac{AUC} se puede calcular a partir de la \ac{ROC}, y es el área bajo la curva formada por ésta. Cuanto mayor sea su valor, más completo y eficaz será el modelo, ya que así se ve cómo aumenta el valor de la \ac{TPR} y cómo decrece el valor de la \ac{FPR}.

Todo este proceso se realiza bajo el contexto de la predicción realizada por el modelo para clasificar las imágenes. Se puede usar tanto para clasificación binaria como para multiclase (\cite{roc-auc-score}). %cita

\section{Modelos}

A continuación vamos a explicar en qué consisten los modelos de Deep Learning usados para la clasificación de imágenes: \textbf{ResNet-34}, \textbf{ResNet-18} y \textbf{AlexNet}.

\subsection{ResNet-34}

\textbf{ResNet-34} es una \ac{CNN} de 34 capas utilizada, en parte, como un modelo de clasificación de imágenes, siendo ésta una de las tecnologías más avanzadas en esta materia (\cite{resnet34}). Este modelo ha sido entrenado bajo el set de imágenes de \textbf{ImageNet} (con más de 100.000 imágenes de 200 clases diferentes en su haber) (\cite{imagenet}). No obstante, y antes de seguir, conviene explicar qué es una \ac{CNN}. %cita %cita

Una \ac{CNN} es una clase de red neuronal formada por capas convolucionales, mayormente usada para el análisis y la clasificación de imágenes. La primera capa extrae información de una imagen, generará una serie de funciones que, finalmente, se acabarán activando para tratar los datos de entrada y el resultado se pasará a la siguiente capa. Este procedimiento sigue para el resto de capas.

Hay un detalle que se debe señalar: cuanto más profunda sea una \ac{CNN} (esto es, cuantas más capas tenga), mayor precisión y mayor detalle habrá para analizar las imágenes (\cite{cnn}). Este dato es muy importante, ya que, como veremos en la siguiente sección entre ResNet-34 y ResNet-18, indicará por qué con uno de estos modelos se ha conseguido una mayor precisión en la clasificación que con el otro. %cita

\subsection{ResNet-18}

\textbf{ResNet-18}, al igual que ResNet-34, es una \ac{CNN} que se compone de 18 capas convolucionales y ha sido preentrenada bajo el set de imágenes de ImageNet (\cite{resnet18}).

En esencia, es una escisión de ResNet-34 que hemos escogido para ver si, usando una \ac{CNN} con menos capas, varía mucho o no la efectividad en la clasificación para un set de imágenes tan pequeño (en comparación con ImageNet).

\subsection{AlexNet}

\textbf{AlexNet} es una \ac{CNN} de 8 capas (5 de ellas convolucionales). En su momento, AlexNet surgió como solución para poder entrenar un modelo de Deep Learning con un set de imágenes muy grande (como ImageNet). En otras palabras, es una arquitectura líder en detección de objetos y con un amplio potencial para poder aplicarse en más ámbitos de visión por ordenador (\cite{alexnet}, \cite{alexnet-2}).

Una característica muy interesante, en comparación con los otros modelos ya vistos, es su gran efectividad para la poca profundidad de capas que lo compone. Además, sabiendo sus precedentes en cuanto a tareas de análisis y clasificación de imágenes, no podíamos no elegirlo como un potencial candidato para cumplir con los objetivos de este trabajo.

En el siguiente punto, pasamos a mostrar los datos arrojados por los modelos para cada métrica.

\newpage

\section{Resultados finales}

A continuación, vamos a mostrar en la siguiente tabla los resultados generados por cada modelo para cada una de las métricas ya explicadas:

\begin{table}[H]
\centering
\resizebox{16cm}{!}{
\begin{tabular}{|c|c|c|c|}\cline{1-4}
\textbf{Modelos} & \textbf{Métricas} & \textbf{Epoch} & \textbf{Resultados}\\ \cline{1-4}
\multirow{4}{3cm}{\textbf{\textit{ResNet-34}}} & \multirow{2}{3cm}{Error Rate} & 0 & 0.049 \\ \cline{3-4}
& & 24 & 0.002 \\ \cline{2-4}
& \multirow{2}{3cm}{Average Precision Score} & 0 & \textbf{0.60} \\ \cline{3-4}
& & 24 & 0.98 \\ \cline{2-4}
& \multirow{2}{3cm}{Balanced Accuracy} & 0 & \textbf{0.92}\\ \cline{3-4}
& & 24 & \textbf{1} \\ \cline{2-4}
& \multirow{2}{3cm}{ROC AUC Score} & 0 & 0.88 \\ \cline{3-4}
& & 24 & \textbf{1} \\ \cline{1-4}
\multirow{4}{3cm}{\textbf{\textit{ResNet-18}}} & \multirow{2}{3cm}{Error Rate} & 0 & \textbf{0.036} \\ \cline{3-4}
& & 24 & \textbf{0} \\ \cline{2-4}
& \multirow{2}{3cm}{Average Precision Score} & 0 & 0.57 \\ \cline{3-4}
& & 24 & \textbf{1} \\ \cline{2-4}
& \multirow{2}{3cm}{Balanced Accuracy} & 0 & 0.80\\ \cline{3-4}
& & 24 & 0.99 \\ \cline{2-4}
& \multirow{2}{3cm}{ROC AUC Score} & 0 & \textbf{0.89} \\ \cline{3-4}
& & 24 & 0.99 \\ \cline{1-4}
\multirow{4}{3cm}{\textbf{\textit{AlexNet}}} & \multirow{2}{3cm}{Error Rate} & 0 & 0.049 \\ \cline{3-4}
& & 24 & 0.014 \\ \cline{2-4}
& \multirow{2}{3cm}{Average Precision Score} & 0 & 0.32 \\ \cline{3-4}
& & 24 & 0.90 \\ \cline{2-4}
& \multirow{2}{3cm}{Balanced Accuracy} & 0 & 0.59 \\ \cline{3-4}
& & 24 & 0.88 \\ \cline{2-4}
& \multirow{2}{3cm}{ROC AUC Score} & 0 & 0.77 \\ \cline{3-4}
& & 24 & 0.99 \\ \cline{1-4}
\end{tabular}}
\caption{Tabla de resultados}
\label{tab:Resul_ISA2}
\end{table}

Los valores que aparecen \textbf{resaltados} indican que son los mejores valores de cada iteración (\textbf{``Epoch''}) de entre todos los modelos para cada una de las métricas.

Conviene destacar algo muy importante: la primera iteración, independientemente de su valor, indica más o menos cómo se va a desempeñar el modelo para las siguientes; sin embargo, lo que realmente importa es ver si en la última iteración ha habido una gran mejoría o, al menos, si se ha conseguido el mejor resultado posible para esa métrica.

\subsection{Error Rate}

En términos de \textbf{``Error Rate''}, el mejor modelo es \textbf{ResNet-18}: tanto para la primera iteración como para la última se obtienen los mejores resultados para esta métrica (\textbf{0.036} y \textbf{0}). Con \textbf{ResNet-34} se obtienen resultados muy parejos con el anterior modelo, sobre todo para la última iteración (\textbf{0.049} y \textbf{0.002}), mientras que con \textbf{AlexNet} hemos conseguido los peores resultados (\textbf{0.049} y \textbf{0.014}).

Un apunte muy interesante es ver cómo con ResNet-34 se obtiene un valor en la última iteración muy parecido al de ResNet-18; recordemos que ResNet-18 es un modelo muy parecido a ResNet-34 pero con menos capas convolucionales, es decir, ResNet-34 es un modelo más ``completo''. Sin embargo, ResNet-18 consigue un mejor resultado en la última iteración (aunque sea por muy poco). Esto no es un indicativo para elegir a ResNet-18 como el mejor modelo, porque aunque tenga mejor resultado, ResNet-34 se queda muy cerca y hay más métricas que valorar. No obstante, es interesante ver cómo dos modelos de la misma familia con diferente número de capas consiguen resultados muy parejos.

\subsection{Average Precision Score}

Para esta métrica, el mejor valor en la primera iteración se obtiene con \textbf{ResNet-34} (\textbf{0.60}), mientras que en la última iteración se obtiene con \textbf{ResNet-18} (\textbf{1}).

Con esta métrica medíamos en un solo valor de 0 a 1 la curva \ac{PR}, por lo que como se puede ver para ambos modelos, se obtienen cifras muy parecidas y realmente buenas que indican que la precisión del modelo (junto con la sensibilidad del mismo) son las mejores que cabrían esperar para el set de imágenes que hemos entrenado.

Por el contrario, con \textbf{AlexNet} no se consiguen datos tan brillantes, pero ello tampoco es indicativo de que sea un mal modelo. Simplemente, no consigue las mismas prestaciones que ambos ResNet.

\subsection{Balanced Accuracy}

Dentro de \textbf{Balanced Accuracy} el mejor modelo sin lugar a dudas es \textbf{ResNet-34}, obteniendo para la primera y última iteración estos valores: \textbf{0.92} y \textbf{1}.

Conviene señalar un punto muy importante para la primera iteración de esta métrica: el valor obtenido nos indica que este modelo puede ser un fuerte candidato a ser el mejor de todos para los objetivos de este trabajo, ya que, como dijimos antes, la primera iteración nos indica cómo se va a desempeñar el modelo para las siguientes iteraciones (prueba de ello es cómo en la última iteración se obtiene el mejor valor posible para esta métrica). Dicho de otra forma, \textbf{ResNet-34} es, para esta métrica y con mucha diferencia, el mejor modelo posible.

Recordemos que esta métrica es muy importante para sets de imágenes no equilibrados, esto es, que hay más elementos de una clase que de otra. De este modo, al estar trabajando con un set de imágenes con tan pocos elementos con exploits en su interior, en comparación con el resto de imágenes, podemos declarar que esta métrica tomará un papel muy importante para decidir qué modelo es el mejor.

\subsection{ROC AUC Score}

Por último, pasamos a observar y comentar los resultados obtenidos para \textbf{ROC AUC Score}. Al igual que con el resto de métricas, los mejores valores oscilan entre los dos modelos más representativos del trabajo: \textbf{ResNet-34} y \textbf{ResNet-18}.

A pesar de que el mejor valor para la primera iteración lo consiguió el modelo \textbf{ResNet-18} con \textbf{0.89}, y de que el mejor valor para la última iteración lo obtenemos con el modelo \textbf{ResNet-34} con \textbf{1}; se podría decir que para esta métrica ambos modelos son los mejores, ya que se obtienen valores realmente muy cercanos y parecidos entre sí para cada una de las iteraciones vistas en la tabla.

A modo de conclusión, para \textbf{ROC AUC Score} lo mejor es utilizar los modelos de la familia ResNet. AlexNet, por el contrario, no ha resultado ser tan prometedor como cabría esperar.

\subsection{Conclusión}

Como ya hemos explicado en el punto anterior, la diferencia entre los modelos \textbf{ResNet-34} y \textbf{ResNet-18} reside en la profundidad de su red neuronal... aun así se puede apreciar cómo ambos modelos son de la misma familia aunque tengan sus diferencias, ya que los valores en dichas iteraciones son muy parecidos en la mayoría de las métricas.

Por otro lado, y como contraste, podemos ver cómo \textbf{AlexNet} ha acabado resultando ser uno de los peores modelos. A pesar de tener buenos resultados, no ha podido compararse con las tan buenas prestaciones servidas por los modelos de ResNet.

En definitiva, podemos aventurarnos a decidir cuál ha sido el mejor modelo de entre todos los experimentos realizados: en nuestra opinión, y viendo los resultados obtenidos, hemos decidido que el mejor modelo es \textbf{ResNet-34}.

Es una decisión un tanto controvertida, porque se ha observado cómo en varias métricas \textbf{ResNet-18} ha podido ser mejor, pero tiene un fundamento: \textbf{ResNet-34} es un modelo que a la larga, puede proporcionar mejores resultados que \textbf{ResNet-18}. Esto es, si entrenamos ambos modelos con un set de imágenes mayor, y con un número de iteraciones más grande, la precisión en la detección de malware será mejor en aquel modelo cuya profundidad, en cuanto a capas convolucionales se refiere, sea mayor; ya que analizará las imágenes con mayor detalle y las clasificará mejor. Todo ello, sin olvidarnos de un motivo no menos importante: con la métrica \textbf{Balanced Accuracy} podemos ver el rendimiento de los modelos, y siendo ResNet-34 el que mejores valores ha obtenido podemos concluir en que este modelo, a la larga, será el apropiado para futuras aplicaciones.

En el siguiente capítulo, finalizaremos este trabajo señalando qué conclusiones e ideas finales hemos sacado en claro.


\chapter{Conclusiones}
\label{ch:conc}

Durante el desarrollo de este proyecto hemos introducido código malicioso en imágenes y hemos utilizado una serie de modelos de Deep Learning para detectar dicho código en las imágenes. Al implementar dichos modelos hemos comprobado, en base a una serie de métricas, cómo de buenos han sido a la hora de la detección del malware y cuál de ellos ha sido el mejor.

En el capítulo \ref{ch:res} ya comprobamos en una tabla comparativa los tres modelos utilizados, y resolvimos finalmente en declarar como mejor modelo \textbf{ResNet-34}. Un factor muy importante para decidirlo fue ver las semejanzas que tenía con el segundo mejor modelo \textbf{ResNet-18} y cómo en la métrica de \textbf{Balanced Accuracy}, ResNet-34 superaba con creces al resto de modelos (\textbf{0.92} y \textbf{1}); dicha métrica, tal y como redactamos anteriormente, resulta ser un factor de gran relevancia para tomar la decisión de elegir a ResNet-34 como el mejor modelo posible de entre todos los valorados.

Cabe recordar de nuevo que \textbf{Balanced Accuracy} sirve para medir la exactitud media de los modelos, es decir, es una forma de medir su rendimiento. Es por ello que la tomamos en tanta consideración para discernir cual es, de entre todos, el mejor modelo.

Por todo ello, hemos concluido en que el modelo adecuado para realizar la detección de malware es \textbf{ResNet-34}, con el cual, a largo plazo, prevemos que se obtendrán unos resultados incluso mejores que los actuales. Con vistas al futuro, hemos pensado las siguientes aplicaciones para esta línea de trabajo

\begin{itemize}
\item Mayor número de imágenes en el set para entrenar y optimizar los resultados del modelo.
\item Buscar y probar con nuevos modelos de Deep Learning para obtener mejores resultados.
\item Probar la detección con nuevas formas de ocultación de malware por esteganografía, usando otras herramientas aparte de Stegosploit y utilizando otro tipo de exploits como código a introducir en las imágenes.
\end{itemize}

La ciberseguridad es un reto que sigue sin resolverse. Cuanto más nos esforzamos en solucionar un problema aparecen otros tres y cada vez cuesta más superarlo. Parece ser algo recurrente, pero gracias a la investigación de líneas de trabajo como esta, poco a poco se consiguen resolver estas incidencias y vulnerabilidades, y así conseguimos asegurar un futuro digital más seguro. Tenemos la esperanza de que con este proyecto, lograremos alcanzar dicho futuro más pronto de lo esperado.



%Prepara la sección de apéncices, si es que se necesita.
%\appendix
%\include{apendice-a}
%\include{fichero-apendice-b}


\backmatter

\nocite{*}



%*****************************
%Sección para la bibliografía
%*****************************

%Posibles estilos de bibliografia
\bibliographystyle{plain}
%\bibliographystyle{abbrvnat}
%\bibliographystyle{klunamed}
%\bibliographystyle{unsrt}

\addcontentsline{toc}{chapter}{{}Bibliografía}

%Toma los datos del fichero bibliografia-pfc.bib
\bibliography{bibliografia-tfm}

%\printindex

\end{document}

%Final de la plantilla.