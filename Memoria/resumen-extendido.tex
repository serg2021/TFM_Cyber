\chapter{Resumen Extendido}
%Debe incluir un resumen del trabajo de un máximo de 5 páginas. Resaltar aspectos fundamentales del desarrollo, los resultados más relevantes y las conclusiones.

Como objetivo principal, en este trabajo se persigue presentar un modelo de Deep Learning que sea capaz de detectar malware oculto en imágenes por esteganografía. Con este modelo pretendemos facilitar la experiencia del usuario en su navegador de confianza desde un enfoque más seguro, sabiendo que los ciberataques utilizados mediante el envío de imágenes maliciosas es un tema que está a la orden del día.

Actualmente, existe una variedad pasmosa de ciberataques tanto a usuarios como a organizaciones. Varios de ellos se basan en el envío de malware y páginas de phishing mediante imágenes y links maliciosos. En este trabajo nos centraremos en buscar una forma de evitar los primeros, usando como móvil las imágenes y como herramienta la esteganografía. Para ello vamos a basarnos en la tecnología Deep Learning como principal solución ante este problema.

De forma resumida, vamos a introducir el malware en un set de imágenes de COCO (\cite{coco}) por esteganografía utilizando la herramienta \textbf{Stegosploit} (\cite{stegosploit}). No se lo introduciremos al set completo, de 5000 imágenes 250 tendrán el malware en su interior; así conseguiremos tener un set de imágenes preparado para poder entrenar una serie de modelos de Deep Learning con los que detectar el malware de forma óptima. %cita %cita

Como ya hemos dicho, utilizaremos Stegosploit como herramienta principal para introducir el código malicioso en las imágenes. Stegosploit es una herramienta que consiste en introducir en la carga útil del archivo de imagen la información que se quiere ocultar. Una imagen se compone de cientos de píxeles, cada uno de los cuales se puede componer de tres canales de color: \textbf{rojo}, \textbf{verde} y \textbf{azul}; cada canal a su vez se compone de 8 bits, de modo que un píxel se puede componer de 24 bits. La herramienta se servirá de estos bits para modificarlos e introducir el exploit de forma que se codifique como un flujo de bits, es decir, el malware se codificará como una consecución de bits y se insertará en los bits de los píxeles que componen la imagen modificando los valores de los bits que lo requieran.

Para saber en qué bits se ha de insertar el código, conviene representar a la imagen con otro enfoque: ya no es un conjunto de píxeles, vamos a ir más allá, ahora es una consecución de ``capas de bits''. Como ya hemos dicho antes los píxeles se componen de canales con 8 bits cada uno, de modo que habrá bits que tengan un mayor peso (es decir, más importancia) en la imagen que otros. Esta diferencia se puede representar 8 capas (una capa por bit), siendo las más bajas las que menor peso tienen. Con Stegosploit el objetivo es insertar el código en la imagen en las capas más bajas, ya que son las que menos información aportan y, visualmente, las que menos destacan. Si se introdujese en las capas más altas se vería que la imagen ha sido alterada y por lo tanto su detección sería instantánea. De esta forma, al modificar los valores de los bits más bajos para componer el código que queremos introducir, la detección a simple vista no es posible, ya que con los valores más bajos tan solo se modifica el valor compuesto del píxel de forma muy sutil y con variaciones muy pequeñas.

Para la detección nos basaremos en modelos capaces de realizar una clasificación por imágenes: esto es, los modelos se basarán en dos tipos de etiquetas para las imágenes (\textbf{``imágenes normales''} e \textbf{``imágenes con exploit''}) y conseguirán clasificar las imágenes del set para éstas. Dichas etiquetas las definimos según el nombre del archivo de imagen: las que comiencen con mayúscula serán imágenes normales y las que empiecen por minúscula serán imágenes con exploit. El objetivo de la clasificación no es otro que conseguir que los modelos aprendan, a nivel de bit, qué imágenes pueden contener información oculta y cuáles no. Cuanto más se entrenen con un número mayor de imágenes, mejor; así lograremos que para cualquier imagen (independientemente del nombre del archivo) los modelos sepan cuáles son maliciosas y cuáles no.

Para empezar vamos a trabajar con Stegosploit. Para trabajar con Stegosploit primero debemos crear un servidor HTTP local que introduzca una cabecera especial en cada respuesta porque, como veremos en el capítulo \ref{ch:stego}, si no lo hacemos no se podría ejecutar la herramienta (\cite{server-http}). Creando dicho servidor y añadiendo una etiqueta HTML al código fuente de la herramienta, podemos hacer uso de la misma (como ya hemos dicho, usaremos 250 imágenes en este paso).

Una vez introducido el exploit en las imágenes pasamos a entrenar una serie de modelos de Deep Learning para realizar la detección, concretamente tres: \textbf{ResNet-34} (\cite{resnet34}), \textbf{ResNet-18} (\cite{resnet18}) y (\cite{alexnet}) \textbf{Alexnet}.

Para trabajar con ellos usaremos \textbf{FastAI} (\cite{fastai}), una librería de Python para entrenar modelos de Deep Learning de forma eficiente, rápida y sencilla; y para trabajar con esta librería, como no tenemos en nuestro haber una \ac{GPU}, utilizaremos los servicios de Google Colab (\cite{google-colab}). Así entrenaremos los modelos.

Para medir la eficacia de los entrenamientos nos serviremos de 4 tipos de métricas: \textbf{Error Rate} (\cite{error-rate}), \textbf{Average Precision Score} (\cite{apscore}), \textbf{Balanced Accuracy} (\cite{balanced-accuracy}, \cite{balanced-accuracy-2}) y \textbf{ROC AUC Score} (\cite{roc-auc-score}).

Según nuestros experimentos, con dos de los tres modelos cotejados se consiguen obtener unos resultados realmente esperanzadores para poder utilizarse en un futuro  (\textbf{ResNet-34} y \textbf{ResNet-18}). Sin embargo, aun teniendo unos valores muy parejos entre sí, finalmente nos decantamos por \textbf{ResNet-34} como estandarte para la realización de trabajos de la misma rama con recursos y objetivos más avanzados.

\begin{table}[H]
\centering
\resizebox{12cm}{!}{
\begin{tabular}{|c|c|c|c|}\cline{1-4}
\textbf{Modelos} & \textbf{Métricas} & \textbf{Epoch} & \textbf{Resultados}\\ \cline{1-4}
\multirow{4}{3cm}{\textbf{\textit{ResNet-34}}} & \multirow{2}{3cm}{Error Rate} & 0 & 0.049 \\ \cline{3-4}
& & 24 & 0.002 \\ \cline{2-4}
& \multirow{2}{3cm}{Average Precision Score} & 0 & 0.60 \\ \cline{3-4}
& & 24 & 0.98 \\ \cline{2-4}
& \multirow{2}{3cm}{Balanced Accuracy} & 0 & 0.92\\ \cline{3-4}
& & 24 & 1 \\ \cline{2-4}
& \multirow{2}{3cm}{ROC AUC Score} & 0 & 0.88 \\ \cline{3-4}
& & 24 & 1 \\ \cline{1-4}
\multirow{4}{3cm}{\textbf{\textit{ResNet-18}}} & \multirow{2}{3cm}{Error Rate} & 0 & 0.036 \\ \cline{3-4}
& & 24 & 0 \\ \cline{2-4}
& \multirow{2}{3cm}{Average Precision Score} & 0 & 0.57 \\ \cline{3-4}
& & 24 & 1 \\ \cline{2-4}
& \multirow{2}{3cm}{Balanced Accuracy} & 0 & 0.80\\ \cline{3-4}
& & 24 & 0.99 \\ \cline{2-4}
& \multirow{2}{3cm}{ROC AUC Score} & 0 & 0.89 \\ \cline{3-4}
& & 24 & 0.99 \\ \cline{1-4}
\end{tabular}
}
\caption{Resultados de ResNet-34 y ResNet-18}
\label{tab:Resul_ResNet}
\end{table}

En la tabla \ref{tab:Resul_ResNet} podemos ver la comparativa entre ambos modelos de la familia ResNet. Como se puede apreciar los dos dan unos resultados muy parecidos entre sí, y en muy pocas métricas se puede ver uno que destaque sobre el otro: en el caso de \textbf{ResNet-18} podemos comprobar que sus valores para con \textbf{Error Rate} son mejores que con \textbf{ResNet-34}; y, análogamente, lo mismo pasa con \textbf{ResNet-34} para la métrica de \textbf{Balanced Accuracy}.

ResNet-34 y ResNet-18 acaban finalmente siendo los mejores modelos, pero hemos de decidir cuál de ellos resultará en una mejor utilización para el futuro. Observando los resultados de la tabla, concluimos con que \textbf{ResNet-34} acaba siendo un mejor modelo, aunque no con demasiada diferencia, por encima de \textbf{ResNet-18}. Lo decidimos así en gran parte por la métrica de \textbf{Balanced Accuracy}, que indica de forma aproximada cómo de bien rinde un modelo de Deep Learning; por otro lado, elegimos ResNet-34 por la profundidad de las capas convolucionales que se utilizan en la \ac{CNN} que lo conforma (\cite{cnn}): a mayor profundidad, mayor detalle y precisión en el análisis de imágenes; y por lo tanto mejores resultados en la clasificación.

Sin embargo, y como indicaremos más adelante, con un set de imágenes como el que usamos no es suficiente para entrenar el modelo y que éste sea capaz de detectar el malware de la imagen a nivel de bit: hacen falta más entrenamientos con sets de imágenes más preparados. Esto, como ya hemos dicho, lo indicaremos más adelante.

Por todo ello, y para concluir, se puede decir que: ResNet-34 es el modelo elegido por nosotros como el mejor candidato para ser entrenado por nuevos sets de imágenes y aplicaciones que persigan los objetivos de este trabajo, pero con una mayor incisión en el problema y tratándolo con herramientas y recursos más avanzados. Hemos utilizado \textbf{Stegosploit} y \textbf{FastAI}, pero estas herramientas tan sólo son la punta del iceberg dentro de un mundo que avanza cada vez más a pasos agigantados.