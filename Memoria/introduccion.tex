\chapter{Introducción}
\label{ch:intro}

Durante muchos años, ciberdelincuentes de todo el mundo se han servido de un gran número de tipos de ataques, generalmente, con la intención de lucrarse a costa de las víctimas o de, directamente, sembrar el caos (aunque puede haber más motivos); todo esto ha sido posible gracias a las vulnerabilidades que presentan los dispositivos informáticos, páginas web... En definitiva, elementos ``hackeables'' (\cite{historia-ciber}).%cita

Sin embargo, y en contraparte, grandes expertos en ciberseguridad y empleados de grandes empresas como Google o IBM han sabido solucionar los problemas que presentan estos dispositivos mediante, por ejemplo, parches en los programas informáticos... No obstante, donde se cierra una puerta se abre una ventana, y las vulnerabilidades corregidas dan paso a otras nuevas potencialmente explotables (\cite{guerra-ciber}).%cita

Y es de esta forma cómo se ha desarrollado este ``diálogo'' entre ambas partes desde, casi, los inicios de Internet inclusive. Sin embargo, con el paso del tiempo se han ido corrigiendo cada vez más cosas, y se está obligando a los ciberdelincuentes a ser más creativos con sus ataques: técnicas de phishing usando ingeniería social (\cite{phishing}), ataques a cadenas de suministro (\cite{cadena-suministro}), archivos con malware introducido en su interior (\cite{malware})... %cita %cita %cita

Cada vez cuesta más, pero poco a poco se van solucionando estos problemas... O al menos eso es lo que pretendemos hacer en este trabajo contribuyendo a la causa y resolviendo un problema que, como veremos, lleva durante mucho tiempo entre los usuarios: los archivos con malware introducido en su interior mediante \textbf{esteganografía}.

\section{Modelo para la detección de malware usando Deep Learning}

Como se puede apreciar en el capítulo \ref{ch:sota}, la mayoría de trabajos y revisiones adoptadas por la comunidad se han basado en Machine Learning y una pequeña parte en Deep Learning. De hecho, en muchos casos, se han obtenido buenos resultados en su detección, pero cabe destacar una observación muy importante: la mayoría de estos métodos basan su detección en información oculta, principalmente, en los \textbf{metadatos} del archivo.

Lo que proponemos con este trabajo es, mediante el uso de Deep Learning, crear un modelo capaz de detectar malware oculto en la misma imagen, es decir, en sus \textbf{datos}. Dicho de otra forma, pretendemos detectar malware oculto en imágenes a nivel de bit de datos.

Esto tampoco es nada nuevo, ya ha sido planteado dentro del capítulo \ref{ch:sota}, sin embargo ha sido desde una perspectiva con vistas a Machine Learning y no a Deep Learning.

El proceso que procederemos a ejecutar más adelante seguirá, grosso modo, la siguiente esquemática:

%IMAGEN

Comenzamos descargando un set de imágenes de Internet. Haciendo uso de la herramienta \textbf{Stegosploit}, de la que hablaremos más adelante, insertaremos malware dentro de los datos de algunas imágenes. Por último, aplicaremos diferentes modelos de Deep Learning para su detección dentro de todo el set. Más tarde, comentaremos los resultados.

\section{Objetivo principal}

El objetivo principal del proyecto es conseguir confeccionar un modelo de detección de malware oculto en imágenes mediante esteganografía, haciendo uso de herramientas de Deep Learning. Como ya hemos dicho anteriormente, los modelos creados para este fin se basan en tecnología Machine Learning, por ello esperamos que este trabajo sea uno de los primeros pasos en el largo camino que queda dentro de esta rama de investigación.

\section{Campos de aplicación}

Este trabajo será aplicable a cualquier ordenador con características apropiadas para la ejecución de tecnología Deep Learning en un tiempo adecuado. De este modo, facilitamos una solución para el uso diario del usuario en su tiempo de ocio, en su horario laboral, para las empresas en el caso de que se reciba por el correo electrónico corporativo una imagen maliciosa... En definitiva, una solución aplicable a una gran variedad de sectores y dominios. 

Dentro de las posibilidades que se presentan, podemos destacar las siguientes aplicaciones:
\begin{enumerate}
\item Implementación de soluciones de detección de malware en imágenes avanzadas.
\item Implementación de nuevas funciones de detección de malware en antivirus.
\item Desarrollo de nuevas tecnologías para facilitar la interacción de los usuarios de forma segura.
\end{enumerate}

Este documento se organiza de la siguiente manera: en el siguiente capítulo hablaremos del estado del arte sobre los sistemas de detección de malware en imágenes para tener una idea clara de los avances que se han ido dando poco a poco durante los últimos años. En el capítulo \ref{ch:det_mal} explicaremos qué es la esteganografía digital (con una breve introducción histórica) y pasaremos a mostrar cómo se estructura nuestro modelo tras la revisión del estado del arte. En el capítulo \ref{ch:stego}, veremos el proceso que hemos seguido para poner en práctica el modelo propuesto, utilizando la herramienta \textbf{Stegosploit} para introducir malware en imágenes (\cite{stegosploit}). Más tarde, en el capítulo de \ref{ch:res}, realizaremos una comparativa del sistema para diferentes modelos de Deep Learning usados en el anterior capítulo, y decidiremos cual es el mejor según las métricas usadas. Por último y para acabar, comentaremos las conclusiones finales de los resultados obtenidos en el capítulo anterior del trabajo en cuestión. %cita